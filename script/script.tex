\documentclass[oneside]{scrbook}
\KOMAoptions{fontsize=11pt, paper=a4}     
\KOMAoptions{DIV=13}                      

\usepackage[utf8]{inputenc}               
\usepackage[T1]{fontenc}                  
\usepackage[varg]{txfonts}  			  %	Times-like fonts in support of mathematics
\usepackage[separate-uncertainty = true]{siunitx}   	  				  
\usepackage{enumitem}				      %	extra enumerate options

\renewcommand{\familydefault}{\rmdefault} % font to sans serif

%import external graphics and where to find these
\usepackage{graphicx}					  
\graphicspath{{figs/}}
% \usepackage{epstopdf}

\RequirePackage[backend=biber, style=numeric]{biblatex}
\addbibresource{intro-integrability.bib}

\usepackage{hyperref}
\RequirePackage[all]{hypcap}

%There are a number of symbols defined inside txfonts that are also defined in amsmath
% so you can just make these available again
\let\iint\relax
\let\iiint\relax
\let\iiiint\relax
\let\idotsint\relax
\let\openbox\relax
\usepackage{amsmath}
\usepackage{amsthm}
\usepackage{physics}
\usepackage{mathtools}
\usepackage{braket}
% \usepackage{slashed} % feynman slash notation
% \usepackage{simplewick} % wick contraction
% \usepackage{tikz}
% \usepackage{tikz-feynman} % feynman diagrams
\usepackage{here} % figure[H]
\usepackage[makeroom]{cancel}
\usepackage[textsize=tiny]{todonotes}

% Externalizing plots
% \usetikzlibrary{external}
% \tikzexternalize[ prefix=tikz-figs/,
                  % mode=list and make,
                  % system call={ lualatex \tikzexternalcheckshellescape -halt-on-error -interaction=batchmode -jobname="\image" "\texsource"  || rm "\image.pdf"},
% ]

%restart footnotes every page
\usepackage{perpage}
\MakePerPage{footnote}
%symbols for footnotes
\usepackage[symbol]{footmisc}
\renewcommand{\thefootnote}{\fnsymbol{footnote}} % footnote mark with special symbols

% toprule and etc.
\usepackage{booktabs}

% Color
\usepackage{xcolor}
%%%%%%%%%%%%%%%%%%%%%%%%%% NEW COMMAND SECTION %%%%%%%%%%%%%%%%%%%%

%define equal
\newcommand{\defeq}{\vcentcolon =} 
\newcommand{\eqdef}{= \vcentcolon}
\newcommand{\euler}{\mathrm{e}}

%Lagrange density
\newcommand{\lag}{\mathcal{L}} 
%Hamiltonian density
\newcommand{\ham}{\mathcal{H}}
\usepackage{mathrsfs}
\newcommand{\hil}{\mathscr{H}}

%identity matrix
\usepackage{dsfont}
\newcommand{\id}{\mathds{1}}

\newcommand{\vecnab}{\pmb{\nabla}}
\newcommand{\vecx}{\pmb{x}}
\newcommand{\vecy}{\pmb{y}}
\newcommand{\veck}{\pmb{k}}
\newcommand{\vecp}{\pmb{k}}
\newcommand{\N}{\mathbb{N}}
\newcommand{\R}{\mathbb{R}}
\newcommand{\Z}{\mathbb{Z}}
\newcommand{\Co}{\mathbb{C}}
\newcommand{\D}{\mathcal{D}}
\newcommand{\M}{\mathcal{M}}
\newcommand{\sm}{Standard Model }
\newcommand{\diag}{\text{diag}}
\newcommand{\sgn}{\text{sgn}}
\newcommand{\cl}{\text{cl}}
\newcommand{\eff}{\text{eff}}
\newcommand{\SU}{\mathbf{SU}}
\newcommand{\SO}{\mathbf{SO}}
\newcommand{\Uni}{\mathbf{U}}
\newcommand{\Deltaop}{\Delta^{\text{op}}}
\newcommand\revvec[1]{\overleftarrow{#1}}

\newtheorem{definition}{Definition}[chapter]
\newtheorem{theorem}{Theorem}[chapter]
\newtheorem{prop}{Proposition}[chapter]
\newtheorem{example}{Example}[chapter]
%%%%%%%%%%%%%%%%%%%%%%%%%%%%% SETTINGS %%%%%%%%%%%%%%%%%%%%%%%%%%%%%%%%%%%%

%%%%%%%%%%%%%%%%%%%%%%%%%%%%%%%%%%%%%%%%%%%%%%%%%%%%%%%%%%%%%%%%%%

\title{Introduction to integrability}
\author{Chenhuan Wang\\
Lecture by Florian Loebert in WS2022/2023}
\date{\today}
\begin{document}
\maketitle
\tableofcontents

\setcounter{chapter}{-1}
\chapter{Preface}
\section{What is integrability?}
There is no universal definition and depends on context and model. However, the harmonic oscillator, which can be approximation of some complete model, is integrable.

\section{FPUT paradox}
In year 1954, Los Almos, USA, there was a brand new computer MANIAC I. Fermi had published an article in 1923 "Beweis dass mechanisches Normalsystem im Allgemein quasiergodisch ist". Here ergodic means equipartion of energy among different degrees of freedom. With this new machine, he did numerical experiment on how long it takes until equipartition is reached. A simple model would be a discrete vibrating string. The Hamiltonian is 
\begin{equation*}
	H(p, q) =\frac{1}{2} \sum_{i=1}^{L}  p_i^2 + \frac{1}{2} \sum_{i=1}^{L} (q_i - q_{i-1})^2 - 2\alpha \sum_{i=1}^{L} (q_i - q_{i-1})^3
\end{equation*}
with the fixed boundary condition $q_0 = q_L = 0$. By defining the normal mode coordinates
\begin{align*}
	Q_k &= \sqrt{\frac{2}{L+1}} \sum_{i=1}^{L} \sin(\frac{ik\pi}{L+1}) q_i, \\
	P_k &= \sqrt{\frac{2}{L+1}} \sum_{i=1}^{L} \sin(\frac{ik \pi}{L+1}) p_i,
\end{align*}
the Hamiltonian becomes
\begin{equation}
	H(P, Q) = \frac{1}{2} \sum \left( P_k^2 + \omega_k^2 Q_k^2 \right)  + \alpha V_e(Q),
\end{equation}
with 
\begin{equation*}
\omega_k = 2 \sin(\frac{k\pi}{2(L+1)})	.
\end{equation*}
The first term in the Hamiltonian is a good approximation for energy per site for small $\alpha$ and the last term is the non-linear term. One can define the average energy per site
\begin{equation}
	\bar{E}_k = \frac{1}{t_0} \int_t^{t_0} E_k^0 (t) \dd{t}
\end{equation}
The expectation (from ergodicity or equipartion) is $\lim_{t\rightarrow \infty} \bar{E}_k(t_0) = \epsilon$ for all $k$.

In the FPUT test, some initial energy is given to mode $k=1$ and wait until equilibrium is reached. The results are shown in figure
\begin{figure}[ht]
	\centering
	\includegraphics[width=0.4\textwidth]{./figs/FPUT-1.png}%
	\includegraphics[width=0.5\textwidth]{./figs/FPUT-2.png}
	\caption{Averaged energy spectrum of normal modes $\bar{E}_k (T)$ plotted against $k/N$ at selected times $T$ and instantaneous values $E_k(t)$ for modes $k=1, 2, 3$ \cite{benettinFermiPastaUlamProblemIts2013}.}
	\label{fig:}
\end{figure}

System shows periodicity instead of equipartition of energy! Why? Are there some hidden symmetries? Still today, there is no satisfactory explanation for FPUT paradox. The FPUT system is close to a class of so-called \textit{integrable models}, whose a number of (hidden) symmetries, roughly speaking, is the same as the degrees of freedom. 

One obtains FPUT model by introducing perturbation to harmonic oscillator (which is integrable). By truncating FPUT model, one has Toda chain. Taking the continuum limit, FPUT model leads to Korteweg-de Vries (KdV) equation, which is integrable. This lecture is about these integrable models with ``large'' number of symmetries, about mathematical formulation and implications.

About literacy ``integrable'': There is possibility to integrate equation of motion to obtain a solution in a form as closed as possible (may require some extra steps).

\chapter{Integrability in classical mechanics}
\section{Hamiltonian Formalism}
Motion of a system with $n$ degrees of freedom described by trajectory in $2$ dimensional phase space $\mathcal{M}$ (manifold) with \textbf{local} coordinates $(p_j, q_j), j=1,\dots, n$.

\paragraph{Dynamical variables} are some function $f: \mathcal{M}\times \R \rightarrow \R, f=f(p, q, t)$.

\paragraph{Poisson brackets}
\begin{equation}
	\left\{f, g  \right\} := \sum_{i=1}^{k} \pdv{f}{q_k} \pdv{g}{p_k} - \pdv{f}{p_k}\pdv{g}{q_k}
\end{equation}
with the properties
\begin{align*}
	\pb{f}{g} &= - \pb{f}{g}  \\
	\pb{f}{\pb{g}{h}} &+ \pb{g}{\pb{h}{f}} + \pb{h}{\pb{f}{g}}  = 0
	\label{eq:}
\end{align*}
and the canonical ``commutation'' relation
\begin{equation*}
	\pb{p_j}{p_k} 	 = \pb{q_j}{q_k}  = 0, \quad \pb{p_j}{q_k}  = \delta_{jk}
\end{equation*}

Given a Hamiltonian $H = H(p, q, t)$, the dynamics of a dynamical variable is determined by 
\begin{equation*}
	\dv{f}{t} = \pdv{f}{t} + \pb{f}{H}	
\end{equation*}
for any $f = f(p, q)$.

Setting $f=p_j$ or $f=q_j$ yields the Hamilton's equation of motion
\begin{equation}
	\dot{p}_j = - \pdv{H}{q_j}, \quad \dot{q}_j = \pdv{H}{p_j}
	\label{math:ham-eom}
\end{equation}
The system \eqref{math:ham-eom} of $2n$ ordinary differential equations (ODEs) is deterministic, meaning the $(p_j(t), q_j(t))$ are uniquely determined by $2n$ initial conditions.

\begin{definition}
	A function $f= f(p_j, q_j, t)$ which $\dot{f} = 0$, when equation of motion \eqref{math:ham-eom} hold, is called	a \textbf{first integral}, a \textbf{constant of motion}, or a \textbf{conserved charge}.
\end{definition}
Equivalently, $f(p(t), q(t), t) = \text{const}$, if $p(t)$ and $q(t)$ satisfy \eqref{math:ham-eom}.

Hamilton's equations will be solvable, if there are ``sufficiently'' many constants of motion.

\paragraph{Example}
	System with one degree of freedom with $\M = \R^2$ and Hamiltonian $H = \frac{1}{2} p^2 + V(q)$. The Hamilton's equations are
	\begin{equation*}
		\dot{q} = p, \quad \dot{p} = - \pdv{V}{q}	.
	\end{equation*}
	The Hamiltonian $H$ is a first integral	 ($\dv{H} = 0$). Thus,
	\begin{align*}
		\frac{1}{2} p^2 &+ V(q) = E = \text{const} , \\
		\dot{q} &= p, p = \pm \sqrt{2(E-V(q))} , \\
		\Rightarrow t &= \pm \int \frac{\dd{q}}{\sqrt{2(E-V(q))}}
	\end{align*}
	Explicit solution could be found if the integral can be performed and the relation $t=t(q)$ can be inverted to get $q(t)$. These two steps are not always possible, but still it is called \textbf{integrable}.

One can also look at the systems \textbf{geometrically}. First integrals defines $f(p, q)=\text{const.}$ in $\M$. Two hypersurfaces corresponding to two first integrals generically intersect in surface of dimension $2$ in $\M$. In general, trajectory lies on a surface of dimension $(2n-L)$ with $L$ the number of independent first integrals. If $L=2n-1$, this ``surface'' is a curve, i.e. a solution to Hamilton's equations.

The questions now is how to find first integrals? If two first integrals are given, their Poisson bracket is another first integral. Noether's theorem gives first integrals (translations, rotations and so on). Energy is always a first integral in Hamilton formalism.

\section{Integrability and action-angle variables}
\begin{definition}
	Consider a Hamiltonian system with $2n$ dimensional phase space $\M$. We call this system (completely) Louville integrable, if $n$ functions $f_1, \dots, f_n: \M \rightarrow \R$ exists such that
	\begin{itemize}
		\item $\pb{f_j}{f_k} = 0, j, k = 1, \dots, n$
		\item $\pb{H}{f_j} = 0, j=1, \dots, n$
		\item The functions $f_1, \dots, f_n$ are independent, i.e. the $\vec{\nabla} f_j$ are linearly independent vectors on a tangent space to any point in $\M$.
	\end{itemize}
\end{definition}
If condition $(1)$ is satisfied, the $f_j$ are in \textbf{involution}. Integrability in the above sense leads to solvability of equation of motion.

\paragraph{Coordinate transformations}
What freedom is there in Hamiltonian structure?

\begin{definition}
	A transformation $Q_k = Q_k(p, q), P_k = P_k(p, q)$ is canonical, if it preserves the Poisson brackets
	\begin{equation*}
		\pb{f}{g}_{p, q} = \pb{f}{g}_{P, Q}, \forall f,g: \M \rightarrow \R.
	\end{equation*}
\end{definition}
Canonical transformation preserves Hamilton's equations. In $2n$ dimensional phase space, only $2n$ of the coordinates $p, q, P, Q$ are independent. Given a generating function $S(q, P, t)$ with
\begin{equation*}
	\det(\pdv[2]{S}{q_i}{p_k}) = 0	
\end{equation*}
we can construct a canonical transformation by setting 
\begin{equation}
	p_k = \pdv{S}{q_k}, \quad Q_k = \pdv{S}{P_k}, \quad H = H+ \pdv{S}{t}
\end{equation}
There are other possibilities with
\begin{align*}
	S(q, Q):& \; p = \pdv{S}{q} , P=-\pdv{S}{Q} , \\
	S(p, Q):& \; P = -\pdv{S}{Q}, q = - \pdv{S}{Q}, \\
	S(p, P):& \; q = - \pdv{S}{p}, Q= \pdv{S}{P}
	\label{eq:}
\end{align*}
Can we find canonical transformation that manifests integrability such that $P_k(t) = P_k(0) =\text{const}$ $n$ constant of motion and $Q_k(t) = Q_k(0) + t \pdv{H}{p_k}$ with linear time dependence. To find such a transformation is in general hard. Deciding whether a given $H$ is integrable is still unsolved problem.

\begin{theorem} (Arnold and Liouville)
	Let $(\M, f_1, \dots, f_n)$ be an integrable system with a Hamiltonian $H=f_1$ and let	
	\begin{equation*}
		\M_f = \left\{ (p, q) \in \M \, |\, f_k (p, q) = c_k = \text{const},\, k=1, \dots, n \right\} 	
	\end{equation*}
	be a so-called $n$ dimensional level set of first integrals $f_n$.
	\begin{enumerate}
		\item if $M_f$ is compact and connected, then it is diffeomorphism to torus $T^n = S^1 \times \cdots \times S^1$.
		\item One introduces (in the neighborhood of this torus in $M$) the action angle variables 
			\begin{equation*}
				I_1, \dots, I_n, \quad \phi_1, \dots, \phi_n, \quad 0 \leq \phi_n \leq 2\pi	
			\end{equation*}
			such that the angles $\phi_k$ are coordinates on $M_f$ and the action (variable) $I_k=I_k(f_1, \dots, f_n)$ are first integrals.
		\item The canonical equations of motion \eqref{math:ham-eom} becomes
			\begin{equation}
				\dot{I}_k = 0, \quad \dot{\phi}_k = \omega_k (I_1, \dots, I_n), \quad k =1, \dots,n 
				\label{math:action-angle-eom}
			\end{equation}
			and the integrable system is solved by \textbf{quadratures} (finite number of algebraic equations and integrations of know functions).
	\end{enumerate}
\end{theorem}

\paragraph{Proof} (not to prove $(1)$ here). On $(2)$ and $(3)$

Motion takes place on surface of dimension $2n -n = n$
\begin{equation}
	f_1 (p,q) = c_1,\; \dots,\; f_n(p, q) = c_n .
	\label{math:arnold-theorem-first-integral}
\end{equation}
From $(1)$, this surface is a torus. Assume $\det(\pdv{f_i}{p_k}) \neq 0$ such that \eqref{math:arnold-theorem-first-integral} can be solved for the momenta $p_i = p_i (q, c)$ with $f_i(q, p(q,c )) = c_i$
\begin{align*}
	\pdv{q_j} & \Rightarrow \pdv{f_i}{q_j} + \sum_{k=0}^{n} \pdv{f_i}{p_k} \pdv{p_k}{q_j} = 0, \\
	 \sum_j \cdot \pdv{f_i}{p_j} &\Rightarrow \sum_j \pdv{f_m}{p_j} \pdv{f_i}{q_j} + \sum_{j, k} \pdv{f_m}{p_j}\pdv{f_i}{p_k} \pdv{p_k}{q_j} = 0, \\
	(mi) - (im) &\Rightarrow \pb{f_i}{f_m} + \sum_{j, k} \left( \pdv{f_m}{p_j} \pdv{f_i}{p_k} \pdv{p_k}{q_j} - \pdv{f_i}{p_j} \pdv{f_m}{p_k} \pdv{p_k}{q_j} \right)  = 0, \\
					&\Rightarrow \sum_{j, k} \pdv{f_i}{p_k} \pdv{f_m}{p_j} \left( \pdv{p_k}{q_j} - \pdv{p_j}{q_k} \right)  = 0, \\
	\left( \pdv{f_i}{p_k} \right) \text{ invertible} &\Rightarrow \pdv{p_k}{q_j} - \pdv{p_j}{q_k} = 0, \\
	\text{Stockes' theorem} &\Rightarrow \oint_{\mathcal{G}} \sum_{j=1}^{n} p_j \dd{q_j} = 0,
\end{align*}
for any closed curve on torus $T^n$ such are contractible to a point. On $T^n$ there are $n$ closed curves that cannot be contracted to a point, such that the corresponding integrals do not vanish.
\todo{Is compactness a condition for Stockes' theorem?}

\begin{definition}
\textbf{action variable}
\begin{equation}
	I_k := \frac{1}{2\pi} \oint_{\Gamma_k} \sum_{i=1}^{n} p_j \dd{q_j}, \quad k = 1, \dots, n
	\label{math:action-variable-def}
\end{equation}
where the curve $\Gamma_k$ is the $k$-th basic cycle on the torus $T^n$
\begin{equation*}
\Gamma_k = \left\{ (\tilde{\phi}_1, \dots \tilde{\phi}_n) \in T^n;\; 0 \leq \tilde{\phi}_k \leq 2\pi, \tilde{\phi}_j = \text{const, for} j\neq k  \right\}.
\end{equation*}

\end{definition}
$\tilde{\phi}_k$ denotes some coordinates on $T^n$. To find these coordinates is non-trivial, in practice it is not clear how to describe a torus explicitly. Arnold-Liouville theorem has character of existence theorem. \todo{why is it non-trivial? In 2d case, they can be parametrised easily.}

Stockes' theorem implies the action variables \eqref{math:action-variable-def} are independent of choice of $\Gamma_k$. The action variable \eqref{math:action-variable-def} are first integrals since $\oint p(q, c) \dd{q}$ only depend on $c_k = f_k$ and $f_k$'s are first integrals.

We have all the action variable in involution, since
\begin{align*}
	\pb{I_i}{I_j} &= \sum_{r, s, k} \left( \pdv{I_i}{f_r} \pdv{f_r}{q_k} \pdv{I_j}{f_s} \pdv{f_s}{p_k} - \pdv{I_i}{f_r} \pdv{f_r}{p_k} \pdv{I_j}{f_s} \pdv{f_s}{q_k} \right),  \\
					  &= \sum_{r,s} \pdv{I_j}{f_r} \pdv{I_j}{f_s} \pb{f_r}{f_s}, \\
					  &=0.
	\label{eq:}
\end{align*}
In particular $\pb{I_k}{H}=0$. 

The torus $\mathcal{M}_f$ can be equivalently defined ny 
\begin{equation*}
	I_1 = \tilde{c}_c, 	\dots, I_n = \tilde{c}_n
\end{equation*}
One may ask why is $I_k$ (as coordinate) better than $f_k$. If one defines $I_k = f_k$, the transformation $(p, q) \rightarrow (I, \phi)$ would not be canonical. 

Canonical angle coordinates $\phi_k$, which are the canonically conjugates to the actions via the generating functions
\begin{equation}
	S(q, I) = \int_{q_0}^{q} \sum_j p_j \dd{q_j},
\end{equation}
with $q_0$ some point on the torus. Modifying $q_0$ just adds a constant to $S$. The angle coordinates are
\begin{equation*}
	\phi_i = \pdv{S}{I_i}.
\end{equation*}
The angles are periodic. Consider two paths $C$ and $C \cup C_k$ (with $C_k = \Gamma_k$ the $k$-th cycle) between $q_0$ and $q$, see figure \ref{fig:torus-angle}.
\begin{figure}[ht]
	\centering
	\includegraphics[width=0.4\textwidth]{./figs/torus-angle.jpg}
	\caption{Torus with two paths $C$ and $C\cup C_k$}
	\label{fig:torus-angle}
\end{figure}
Then
\begin{align*}
	S(q, I) &= \int_{C\cup C_k} \sum_j p_j \dd{q_j} \\
			  &= \int_{C} \sum_j p_j \dd{q_j} + \int_{C_k = \Gamma_k} \sum_j p_j \dd{q_j} \\
			  &= S(q, I) + 2\pi I_k \\
	\Rightarrow \phi_k &= \pdv{S}{I_k} = \phi_k + 2\pi
	\label{eq:}
\end{align*}

The transformation
\begin{equation*}
	q = q(\phi, I), \quad p = p(\phi, I)
\end{equation*}
and 
\begin{equation*}
	\phi = \phi(p, q) ,\quad I = I(p, q)
\end{equation*}
are canonical transformations (defined by the generating function $S$) and invertible. The Poisson structures are unchanged
\begin{align*}
	\pb{I_j}{I_k} = 0, \quad \pb{\phi_j}{\phi_k} = 0, \quad \pb{\phi_j}{I_k} = \delta_{jk}
	\label{eq:}
\end{align*}
The dynamics are given by
\begin{align*}
	\dot{\phi}_k = \pb{\phi_k}{\tilde{H}}, \quad \dot{I}_k = \pb{I_k}{\tilde{H}}
\end{align*}
with $\tilde{H} = \tilde{H}(\phi, I) = H(q(\phi, I), p(\phi, I))$. Since $I_k$'s are first integrals, 
\begin{equation*}
	0 = \dot{I}_k = \pdv{\tilde{H}}{\phi_k} ,
\end{equation*}
in other word $\tilde{H} = \tilde{H}(I)$. The derivatives of angle variable
\begin{equation*}
	\dot{\phi}_k = \pdv{\tilde{H}}{I_k} = \omega_k (I)
\end{equation*}
are first integrals as well.

Integration (``integrable'' model) yields
\begin{align}
	\begin{split}
		\phi_k (t) &= \omega_k(I) t + \phi_k (0), \\
		I_k(t) &= I_k(0).
	\end{split}
	\label{math:action-angle-sol}
\end{align}
The system is in a circular motion with constant angular velocity.

\paragraph{Geometric picture}
The phase space of an integrable system  is  foliated into an $n$-parameter ($c_j$) family of invariant tori on which flow is linear with constant frequency $\omega_k$. The trajectory \eqref{math:action-angle-sol} may be closed on the torus or it may cover it densely. For $n=2$, the trajectory is closed if $\omega_1/\omega_2$  is rational and dense otherwise.

\paragraph{Degeneracy}
The periodicity in $\phi$ means that every function $F(p, q)$ of the state of system is periodic in $\phi$. Expand the function in Fourier series, e.g. $n=2$
\begin{align*}
	F &= \sum_{l_1 = -\infty}^{\infty}\sum_{l_2 = -\infty}^{\infty} B_{l_1, l_2} \exp(i (l_1 \phi_1 + l_2\phi_2)), \\
	  &= \sum_{l_1, l_2} B_{l_1, l_2} \exp(it(l_1 \omega_1 + l_2 \omega_2)).
\end{align*}
Every summand is period with frequency $l_1\omega_1 + l_2\omega_2$. Sum of functions is not necessarily periodic. The whole sum is only periodic for rational $\omega_1 / \omega_2$. If $a_j \omega_j = a_k \omega_k$ for $a_{j, k} \in \Z$ for some $j, k$, one speaks of \textbf{degeneracy}. If $a_1\omega_1 = \dots = a_n \omega_n$, the system is \textbf{maximally} degenerate.

\paragraph{Example}
All time-indepedent Hamiltonian systems with $2$ dimensional phase-space are integrable ($H=f_{1=n}$). 

Consider a harmonic oscillator ($n=1$) with the Hamiltonian
\begin{equation*}
	H = \frac{1}{2} (p^2 + \omega^2 q^2)
\end{equation*}
Different choices of energy $c_1 = E$ give foliation of $\M$ by ellipses
\begin{equation*}
	\frac{1}{2} \left( p^2 + \omega^2 q^2 \right) 	 = E
\end{equation*}
with two axes $a = \sqrt{2E}$ and $b = \frac{\sqrt{2E}}{\omega}$ and surface $ab\pi$. For fixed $E$, take $\Gamma = \M_H$
\begin{equation*}
	I = \frac{1}{2\pi} \oint d\dd{q} \stackrel{\text{Stockes'}}{=} \frac{1}{2\pi} \int_{S} \dd{p} \dd{q} = \frac{E}{\omega}
\end{equation*}
The Hamiltonian in the new variable $\tilde{H} = \omega I$ and $\dot{\phi} = \pdv{\tilde{H}}{I} = \omega$, $\phi = \omega t + \phi_0$. 

To obtain the transformation $(p, q) \rightarrow (I, \phi)$, first the action variable is
\begin{equation*}
	I(p, q) = \frac{1}{\omega}H(p, q) = \frac{1}{2} \left( \frac{1}{\omega} p^2 + \omega q^2 \right) .
\end{equation*}
The generating function is
\begin{equation*}
	S(q, I ) = \int_{q_0}^{q} p \dd{\tilde{q}} = \pm \int_{q_0}^{q} \sqrt{2 I \omega - \omega^2 \tilde{q}} \dd{\tilde{q}}
\end{equation*}
and the angle variable
\begin{equation*}
	\phi = \pdv{S}{I} = \int \frac{\omega\ \dd{\tilde{q}}}{\sqrt{2I\omega - \omega^2 \tilde{q}^2}} = \arcsin(q \sqrt{\frac{\omega}{2I}}) - \phi_0.
\end{equation*}
Thus
\begin{align*}
	q &= \sqrt{\frac{2 E}{\omega}} \sin(\omega t + \phi_0) \\
	p &= \pdv{H}{p} = \dot{q} = \sqrt{2E} \cos(\omega t + \phi_0)
\end{align*}


%%%%%%%%%%%%%%%%%%%%%%%%%%%%%%%%%%%%%%%%%  lecture 3
\paragraph{Example} The Kepler Problem ($n=2$)

Consider the Motion in two-dimensional phase space (reduced from three-dimensional to two-dimensional using angular momentum conservation). Then we have four dimensional phase space $q_1 = \phi, q_2 = r, p_1 = p_\phi, p_2 = p_r$. The Hamiltonian is
\begin{equation*}
	H = \frac{p_\phi^2}{2r^2 } + \frac{p_r^2}{2} - \frac{\alpha}{r}
\end{equation*} 
with a positive constant $\alpha$. We have $\pb{H}{p_\phi} = 0$, the system is (Liouville) integrable ($2$ constants of motion). 

Level set $\M_f: H=E; p_\phi=\mu$. Then we can solve for $p_r$
\begin{align*}
	p_r = \pm \sqrt{2E - \frac{\mu^2}{r^2} + \frac{2\alpha}{r}}.
\end{align*}
$\phi$ is arbitrary, one constraint on $(r, p_r)$. Parametrize $\M_f$ by $\phi$ and function of $(r, p_r)$. Vary $\phi$ and fix other coordinate, consider one cycle $\Gamma_\phi \subset \M_f$
\begin{align*}
	I_\phi &= \frac{1}{2\pi} \oint_{\Gamma_\phi} \left( p_r \dd{r}+p_\phi \dd{\phi}  \right), \\
			 &= \frac{1}{2\pi} \int_0^{2\pi} p_\phi \dd{\phi} = \mu,
\end{align*}

To find the second action, fix $\phi$
\begin{align*}
	I_r &= \frac{1}{2\pi} \oint_{\Gamma_r} p_r \dd{r},	 \\
			 &= 2 \cdot \frac{1}{2\pi}\int_{r_-}^{r_+} \sqrt{2E - \frac{\mu^2}{r^2} + \frac{2\alpha}{r}} \dd{r},
\end{align*}
where we have taken the positive and negative roots and integrate $r_- \rightarrow r_+$ and backwards. Turning points $r_\pm$ are solutions of $2E - \frac{\mu^2}{r^2} + \frac{2\alpha}{r} = 0$ ($p_r \in \R$). Integral can be done using residual calculus 
\begin{equation*}
	I_r = \alpha \sqrt{\frac{1}{2 |E|}} - \mu = \alpha \sqrt{\frac{1}{2 |E|}} - I_\phi.
\end{equation*}
Thus, the Hamiltonian written in terms of actions is
\begin{align*}
	\tilde{H} &= - \frac{\alpha^2}{2(I_r + I_\phi)^2}, \\
	\Rightarrow \pdv{\tilde{H}}{I_r} &= \pdv{\tilde{H}}{I_\phi} = \frac{\alpha^2}{(I_r + I_\phi)^3}.
\end{align*}
This is a particular case with $\omega_r = \omega_\phi$, and therefore closed orbits.


\paragraph{Superintegrability} 
One may wonder why $\tilde{H} = \tilde{H}(I_r + I_\phi) $. Is there some special property in the system unexplored?  In general, an integrable system admits $n$ independent actions $I_k$, that can be uniquely expressed as functions of the system's state. We may write $(n-1)$ additional constants of motion as 
\begin{equation*}
	A_{ik} := \phi_i \pdv{H}{I_k} - \phi_k \pdv{H}{I_i}
\end{equation*}
remember $\dot{\phi_k} = \pdv{H}{I_k} = \omega_k$. Since $\phi_k = \phi_k + 2\pi$, the $A_{ik}$'s are not unique functions.

Suppose we have a degenerate system, e.g.
\begin{equation}
	a_1 \pdv{H}{I_2}	= a_2 \pdv{H}{I_2}
	\label{math:a_1_a_2}
\end{equation}
for $a_1, a_2 \in \Z$. Then 
\begin{equation*}
	B_{12} := a_1 \phi_1 - a_2 \phi_2,
\end{equation*}
is a constant of motion with $B_{12} = B_{12} + 2\pi n, n\in \Z$. Any trigonometric function of $B_{12}$ is unique constant of motion. Here \eqref{math:a_1_a_2} implies $H=H(a_2 I_1 + a_1 I_2)$. For the Kepler problem, the additional symmetry is the well-known \textbf{Laplace-Runge-Lenz vector}.

\begin{definition}
	A Hamiltonian system with $2n$-dimensional phase space and more than $n$ independent constants of motion is called \textbf{superintegrable}. If the system has $2n-1$ independent constants of motion, it is \textbf{maximally superintegrable}.
\end{definition}

\section{Poisson Structures}
Consider phase space $\M$ of dimension $m$ with local coordinates ($\xi^1, \dots, \xi^n$), where we make no distinction between coordinates and momenta.

\begin{definition}
	A skew-symmetric matrix $\omega^{ab} = \omega^{ab} (\xi)$	 is called a \textbf{Poisson structure}, if the Poisson bracket defined by 
	\begin{equation*}
		\pb{f}{g} =  \sum_{a, b = 1}^m \omega^{ab} (\xi) \pdv{f}{\xi^a} \pdv{g}{\xi^b}
	\end{equation*}
satisfies  $\pb{f}{g} = - \pb{g}{f}$ and the Jacobi identity.
\end{definition}
The Jacobi identity puts restrictions on $\omega^{ab}(\xi) = \pb{\xi^a}{\xi^b}$
\begin{align*}
	\sum_{d=1}^{m} \left( \omega^{dc} \pdv{\omega^{ab}}{\xi^d} + \omega^{db} \pdv{\omega^{ca}}{\xi^d} + \omega^{da} \pdv{\omega^{bc}}{\xi^d} \right) 	 = 0
\end{align*}

Given a Hamiltonian $H: \M \times \R \rightarrow \R$, the dynamics is given by
\begin{equation*}
	\dv{f}{t} = \pdv{f}{t} + \pb{f}{H}
\end{equation*}
and the Hamilton's equations generalizing \eqref{math:ham-eom} 
\begin{equation*}
	\dot{\xi}^a = \sum_{b=1}^m \omega^{ab}(\xi) \pdv{H}{\xi^b}
\end{equation*}

\paragraph{Example}
$\M = \R^3, \omega^{ab} = \sum_{c=1}^{3} \epsilon^{abc} \xi^c$, then
\begin{equation*}
	\pb{\xi^a}{\xi^b} = \epsilon^{abc} \xi^c = \sum \epsilon^{abc} \xi^c
\end{equation*}
This Poisson structure admits a Casimir, namaly any function $f(r)$, where
\begin{equation*}
	r = \sqrt{(\xi^1)^2 + (\xi^2)^2 + (\xi^3)^2},
\end{equation*}
and Poisson-commutes with the coordinate function $\pb{f(r)}{\xi^a}=0$.

\paragraph{Symplectic structures}
Assume $m=2n$ even and $\omega$ invertible with $W := \omega^{-1}$. Jacobi identity implies
\begin{equation*}
	\partial_a W_{bc} + \partial_c W_{ab} + \partial_b W_{ca} = 0,\quad \forall a, b, c = 1, \dots, m
\end{equation*}
In this case we call $W$ a \textbf{symplectic structure}.

The \textbf{Darboux theorem} states that there exists locally coordinate system with 
\begin{equation*}
	\xi^1 = q_1, \dots, \xi^n = q_n, \xi^{n+1} = p_1, \dots, \xi^{2n} = p_n
\end{equation*}
such that 
\begin{equation*}
	\omega = \begin{pmatrix} 0 & \id_n \\ -\id_n & 0 \end{pmatrix}
\end{equation*}
and the Poisson bracket reduces to the standard form.

\paragraph{Example} (Spinning Euler Top)

The coordinates are just the angular momentum $\xi^{1, 2, 3} = S^{x, y, z}$. The Hamiltonian is 
\begin{equation*}
	H = \frac{1}{2}\left[ \frac{(S^x)^2}{\Omega_x} + \frac{(S^y)^2}{\Omega_y} + \frac{(S^z)^2}{\Omega_z}\right] 
\end{equation*}
for angular momentum vector $\vec{S}$ of a rigid spinning body fixed at center of mass. $\Omega_i$ is the diagonal entry of moment of inertia matrix $\Omega$, such that $\vec{S} = \Omega \vec{\omega}$ with $\vec{\omega}$ the angular velocities. 

Hamilton's equations (Euler equations) are
\begin{equation*}
	\dv{t} \vec{S} = - \pb{H} {\vec{S}} = \left( \Omega^{-1} \vec{S} \right)  \times \vec{S}.
\end{equation*}
(Sometimes written as three decoupled differential equations.)

There are two conserved charges $H$ and $|\vec{S}|$. Use $H = E$ and $|\vec{S}|=l$ to write the equation of motion as 
\begin{equation*}
	\dv{t} S_x = \sqrt{A + B S_x^2 + C S_x^4}
\end{equation*}
with $A, B, C$ functions of $\Omega, l , E$. Extinguish three cases
\begin{table}[ht]
	\centering
	\label{tab:label}
	\begin{tabular}{c c c c}
	\toprule
	solution & rational & trigonometric & elliptic \\
	\midrule
	name & x x x & x x z & x y z \\
	$\Omega_{x, y, z}$ & $\Omega_x, \Omega_x, \Omega_x$ & $\Omega_{x}, \Omega_x, \Omega_z$ & $\Omega_x, \Omega_y, \Omega_z$ \\
	symmetry & $\SO(3)$ & $\SO(2)$ & - \\
	\bottomrule
	\end{tabular}
\end{table}

\section{Classical chains and Fields}
We can align elementary mechanical models on a one dimensional lattice and it yields a chain model. Examples are FPUT, Toda chain, classical spin chain. Infinite chains have infinitely many degrees of freedom. The question is how many (conserved) charges do we need for integrability? Precise meaning of integrability is not clear, won't discuss classical field here.

\paragraph{Continuum limits} 
Field theories are naturally understood as continuum limits of lattice models (chain). A well-behaved continuum limit of an integrable lattice model should be integrable.

General idea is to consider one dimensional classical chain model of variable $\xi_j$. Sites are labelled by $j$ at position $x_j = x + j\cdot a$ with a constant lattice spacing $a = x_{j+1}  - x_j$. Continuum limit $a\rightarrow 0$. Fix limiting continuous field $\phi(x)$ via 
\begin{equation*}
	\xi_j = X_j (a, \phi(x_j))
\end{equation*}
with $X_j$ some function specifies the limit prescription (There is no well-defined continuous limit!).

The Simplest example is $\xi_j = \phi(x_j)$. In the limit $a\rightarrow 0$ for instance 
\begin{align*}
	\phi'(x) &= \lim_{a\rightarrow 0} \frac{\phi(x_{j+1}) - \phi(x_j)}{a}, \\
	\phi''(x) &= \lim_{a\rightarrow 0} \frac{\phi(x_{j+1}) - 2 \phi(x_j) + \phi(x_{j-1})}{a^2}, \dots
\end{align*}
Lattice sums turn into integrals
\begin{equation*}
	\lim_{a\rightarrow 0} a \sum_j (\dots) = \int \dd{x} \dots .
\end{equation*}
Delta function becomes kronecker delta
\begin{equation*}
	\delta(x-y) = \lim_{a\rightarrow 0} \frac{1}{a} \delta_{jk}
\end{equation*}
May have expressions including different points $x_j = x+ja$ and $x_k = y+ka$, such that 
\begin{equation*}
	x = \lim_{a\rightarrow 0} (x_0 + ja), y = \lim_{a\rightarrow 0} (x_0 + ka)
\end{equation*}
Definition of integrability for field theory is even worse than chain models.


%%%%%%%%%%%%%%%%%%%%%%%%%%%%%%%%%%%%%%%%%%%%%%% lecture 4

\paragraph{Hamiltonian formalism for fields}
Formally, replace coordinates $\xi (t)$ by field variable $\phi(t, x)$, replace phase space $\M(=\R^m)$ by space of smooth functions on a line $(=\R)$ with some boundary conditions (e.g. decay, open, periodic). 
\begin{table}[htpb]
	\centering
	\begin{tabular}{cc}
		\toprule
		elementary mechanics & fields \\
		\midrule
		$\xi^a(t), a = 1, \dots, m$ & $\phi(t, x), x\in \R$ \\
		$\sum_a$ & $\int_{\R}\dd{x}$ \\
		function $f(\xi)$ & functional $F[\phi]$ \\
		$\pdv{\xi^a}$ & $\fdv{\phi}$ \\
		ODEs $(t)$ & PDEs $(t, x)$ \\
		\bottomrule
	\end{tabular}
\end{table}

Functionals given by integrals
\begin{equation*}
	F[\phi] = \int_{\R} f(\phi, \phi_x, \phi_{x x }, \dots) \dd{x}
\end{equation*}
Recall 
\begin{equation*}
	\fdv{F}{\phi(x)} = \pdv{f}{\phi} - \pdv{x} \pdv{f}{\phi_x} + \pdv[2]{x} \pdv{f}{\phi_{x x}}
\end{equation*}
and 
\begin{equation*}
	\fdv{\phi(y)}{\phi(x)} = \delta(x-y)
\end{equation*}
with $\int_{\R} \delta(x) \dd{x} = 1$.

\begin{definition}Poisson bracket in this case can be defined as 
	\begin{equation*}
		\pb{F}{G} = \int_{\R} \omega(x, y, \phi) \fdv{F}{\phi(x)} \fdv{G}{\phi(y)} \dd{x} \dd{y}
	\end{equation*}
with $\omega$ such that the Poisson bracket is anti-symmetric and obeys Jacobi-identity.
\end{definition}

Canonical choice
\begin{equation*}
	\delta(x, y, \phi) = \frac{1}{2} \partial_x \delta(x-y) - \frac{1}{2} \partial_y \delta(x-y)
\end{equation*}
It is analogous to Darboux form, where $\omega$ is constant and anti-symmetric. Antisymmetry is analogous to $\pdv{x}$ being anti-self-dual with respect to inner product $\langle \phi, \psi \rangle = \int_{\R} \phi(x) \psi(x) \dd{x}$. 

Hence, the canonical bracket is 
\begin{equation*}
	\pb{F}{G} = \int_{\R} \fdv{F}{\phi(x)} \pdv{x} \fdv{G}{\phi(x)} \dd{x}
\end{equation*}
with Hamilton's equations $\pdv{\phi}{t} = \pb{\phi}{H[\phi]}$.

\chapter{Inverse scattering method and solitons}
Our previous definition of (Liouville) integrability works for ODEs. No universal definition of integrability for PDEs. One of the problems is that the phase space is infinitely dimensional but having infinitely many first integrals may not be enough (need to compare these infinities). Focus on properties of solutions and solution techniques.

\section{The KdV equation}
John Scott Russell (1808-1882) made experiments  and find efficient design canal boats. His famous quote from Russell's ``Report on Waves'' (1844). Wave in shallow water described by Korteweg-de Vries (KdV) equations
\begin{equation}
	\phi_t - 6 \phi \phi_x + \phi_{x x x} = 0, \quad \phi = \phi(t, x)
	\label{math:KdV}
\end{equation}
which is written down and solved by simplest case in 1895 by KdV to explain Russell's observation.

\paragraph{Physical motivation for KdV equation}
Start with linear wave equation 
\begin{equation*}
	\psi_{x x} - \frac{1}{v^2} \phi_{tt} = 0
\end{equation*}
with the velocity $v$. One can make three assumptions
\begin{enumerate}
	\item invariance in $t \rightarrow -t$
	\item small amplitude; omit terms of order $\psi^2$
	\item constant group velocity; no dispersion
\end{enumerate}
Relax assumptions to obtain arrive at KdV.

Consider general solutions of wave equation
\begin{equation*}
	\psi(t, x) = f(x-vt) + g(x+vt)
\end{equation*}
where $f$ and $g$ can be arbitrary functions. These functions are each characterised by first order PDE, for example
\begin{equation*}
	\psi_x + \frac{1}{v} \psi_t = 0
\end{equation*}
with $\psi = f(x-vt)$.

\paragraph{Introduce dispersion} Consider complex wave $\psi = e^{i(kx - \omega(k)t)}$ with $\omega(k) = vk$. The group velocity $\dv{\omega}{k} $ equals phase velocity $v$.

Modify relation to introduce dispersion
\begin{equation*}
	\omega(k) = v(k - \beta k^3 + \dots)
\end{equation*}
and higher order terms in $k$ are negligible for small dispersion. Quadratic term leads to complex solution, undesirable.

The function $\psi = e^{i(kx - v(kt - \beta k^3 t))}$ satisfies the differential equation
\begin{equation*}
	\psi_{x} + \beta \psi_{x x x} + \frac{1}{v} \psi_t = 0
\end{equation*}
Rewrite this as a conservation law
\begin{equation*}
	\rho_t + j_{x} = 0
\end{equation*}
if we identify
\begin{equation*}
	\rho = \frac{1}{v} \psi, \quad j = \psi + \beta \psi_{x x}
\end{equation*}

\paragraph{Introduce non-linearity} Modify current
\begin{equation*}
	j = \psi + \beta \psi_{x x} + \frac{\alpha}{2} \psi^2 
\end{equation*}
Then we have 
\begin{equation*}
	\frac{1}{v} \psi_t + \psi_x + \beta \psi_{ x x x} + \alpha \psi \psi_x = 0
\end{equation*}
The constants $(v, \beta, \alpha)$ can be eliminated by change of variables (e.g. linear combination of $x$ and $t$) and rescaling and one obtains the KdV equation \eqref{math:KdV}.

The simplest one-soliton solution found by KdV (1895) is 
\begin{equation}
	\phi (t, x) = - \frac{2\chi^2}{\cosh^2[\chi(x-4\chi^2t - \phi_0)]}
\end{equation}
where $\phi_0$ location of extremum at $t=0$ and $\chi\in\R$ a free parameter.

Note the function $c \cdot \phi$ with $c=\text{const}$ is no solution due to the non-linearity.

\paragraph{Numerical evidence for special properties of KdV}
Until 1965 equation \eqref{math:KdV} was the only regular solution $(\phi, \phi_x \stackrel{|x| \rightarrow \infty}{\rightarrow} 0)$. Zabusky and Krusal (1965) observed numerically that two waves scatter without changing their shape. This is particle-like behaviour, thus the name ``soliton'', i.e. solitary-ons (like electrons and so on). The existence of stable solitary wave is a consequence of cancellation between dispersion and non-linearity.

Without dispersion $\phi_t - 6 \phi \phi_x = 0$ has the solution with discontinuity of first derivative at some $t_0 > 0$. Without non-linearity $\phi_t + \phi_{x x x} = 0$, then the wave will disperse. Only with both terms, we would have stable solutions.

\section{Inverse scattering method (ISM)}
The ISM to solve classical soliton equations comes from quantum mechanics.

\paragraph{Mathematical framework for QM}
Infinite-dimensional complex vector space $\mathcal{H}$ of functions. Wave functions $\psi \in \mathcal{H}, \psi: \R \rightarrow \C, \psi = \psi(x)$ (time independent). Inner product defined as 
\begin{equation}
	\lable \psi_1, \psi_2 \rangle = \int_{\R} \bar{\psi}_1 (x) \psi_2(t) \dd{x}
\end{equation}
Bound states are functions with $\langle \psi, \psi \rangle < \infty$ , e.g. $e^{-x^2}$.  Scattering states not square integrable, e.g. $e^{-ix}$.

Given a (real ) potential $\phi = \phi(x)$, the time-indepdent Schrödinger equation (SE) reads 
\begin{equation*}
	- \frac{\hbar^2}{2m} \dv[2]{\psi}{x} + \phi \psi = E \psi
\end{equation*}
and represents eigenvalue problem. Given $\phi(x)$ one can solve the SE.

Physical needs are typically different: measure scattering process/data, i.e. reflection and transmission coefficients and try to recover the potential from it. Now the problem is to recover potential from scattering data.

In 1950s, solved by Delfandm, Levitan, Marchenko (GLM) using algorithm. 1967 Gardener, Greene, Krusal, Miura used that algorithm to solve the Cauchy problem for KdV.

In scattering theory, Determine reflection ($R$) and transimission ($T$) coefficients with continuous energies. Bound state has discrete energy levels ($E$).

GLM method knowledge of $(E, T, R)$ allows to relate the scattering data to the potential. Cauchy problem for KdV with some initial condition $\phi(0, x) = \phi_0 (x)$, in order to get $\phi(t, x)$. Instead, using Schrödinger equation, input scattering data at $t_0$, one get scattering data at $t > 0$. Using GLM integral equation, $\phi(t, x)$ can be computed. This is \textbf{inverse scattering method}.

%%%%%%%%%%%%%%%%%%% lecture 6
\chapter{First integrals and Zero curvature representation}
\section{First integrals and Hamilton's formalism}
We want to make contact to Liouville integrability for infinite dimensional systems. Remember 
\begin{equation*}
	\psi_2 (k, x) = 
	\begin{cases}
		e^{-ikx} & x \rightarrow - \infty \\
		a(k, t) e^{-ikx} + b(k, t) e^{ikx} & x \rightarrow + \infty
	\end{cases}
\end{equation*}
Time evolution of scattering data \eqref{math:2.10} gives $\pdv{t} a(k, t)  = 0$ for all $k$. It means that the scattering data gives infinitely many first integrals, provided they are nontrivial and independent.

One can indeed construct the first integrals
\begin{equation*}
	I_n [\phi] = \int_{\R} P_n(\phi, \phi_x, \phi_{x x}, \dots) \dd{x}
\end{equation*}
with some polynomials $P_n$ and $\dv{t} I_n = 0$. For example, the momentum
\begin{equation*}
	I_0 = \int \frac{1}{2} \phi^2 \dd{x}
\end{equation*}
and energy
\begin{equation*}
	I_1 = -\frac{1}{2} \int (\phi_x^2 + 2 \phi^3) \dd{x}
\end{equation*}
$I_0$ and $I_1$ are associated via Noether's theorem with translation invariance of KdV system.

It can be shown that these conserved quantities are in involution: $\pb{I_m}{I_n} = 0$ with the canonical Poisson bracket
\begin{equation*}
	\pb{F}{G} = \int \fdv{F}{\phi(x)} \pdv{x} \fdv{G}{\phi(x)} \dd{x}
\end{equation*}
c.f. Liouville integrability. Note that the choices of conserved quantities and Poisson structures are not unique.

\section{Zero curvature representation}
Integrable systems are compatibility conditions of overdetermined systems of matrix PDEs.

Let $U(u)$ and $V(u)$ be matrix-valued functions of $(t, x)$ depending on the auxiliary spectral parameter $u$. Consider system 
\begin{equation*}
	\pdv{F}{x} = U(u) F, \quad \pdv{F}{t} = V(u) F
\end{equation*}
with $F$ a vector and $F = F(t, x, u)$. It is overdetermined system: there are twice as many equations as unknowns. Compatibility conditions from cross-differentiation are
\begin{align*}
	&(\partial_t \partial_x - \partial_x \partial_t )F, \\
	\Rightarrow \quad & \partial_t \left(U(u) F\right) - \partial_x \left(V(u) F\right) = \left[ \partial_t U(u) - \partial_x V(u) + \comm{U(u)}{V(u)} \right]  F = 0.
\end{align*}
Thus, the zero-curvature condition is 
\begin{equation}
	\boxed{
	\left[ \partial_t U(u) - \partial_x V(u) + \comm{U(u)}{V(u)} \right]  F = 0
	}
	\label{math:zero-curv}
\end{equation}

Most non-linear integrable equations admit a zero curvature representations.

\begin{example} (sine-Gordon equation)
	Consider the following functions
\begin{equation*}
	U = \frac{i}{2} \begin{pmatrix} 2u & \phi_x \\ \phi_x & -2u \end{pmatrix}, \quad
	V = \frac{1}{4iu} \begin{pmatrix} \cos{\phi} & -i\sin{\phi} \\ i \sin{\phi} & -\cos{\phi} \end{pmatrix},
\end{equation*}
with $\phi = \phi(t,x)$. With the zero-curvature equation \eqref{math:zero-curv}, one has the sine-Gordon equation
\begin{equation*}
	\phi_{xt} = \sin{\phi}.
\end{equation*}
\end{example}

\subsection{From Lax to zero curvature representation}
Goal is to understand Lax equation as compatibility condition. Consider eigenfunction $f$ of the Lax operator $L$  with eigenvalue $E = u$. Then the equation \eqref{math:2.8} becomes
\begin{equation*}
	(L-E) (f_t + Mf) = 0.
\end{equation*}
For a simple eigenvalue $E=u$
\begin{equation*}
	f_t + Mf = c(t) f.
\end{equation*}

It has been shown in the exercise that
\begin{equation}
	\exists \hat{f}: L \hat{f} = u \hat{f}, \quad  \pdv{\hat{f}}{t} + M \hat{f} = 0,
	\label{math:3.3}
\end{equation}
with $\hat{f} = \hat{f}(t, x, u)$.

Start with overdetermined system \eqref{math:3.3} for a Schrödinger operator $L$ and some differential operator $M$. Lax equation is the compatibility condition
\begin{equation*}
	L(\partial_t + M) = (\partial_t + M) L \Rightarrow \dot{L} = \comm{L}{M}
\end{equation*}
(to be understood as acting on a test function.)

Consider a general scalar Lax pair
\begin{align*}
	L = \pdv[n]{x} + a_{n-1}(t, x) \pdv[n-1]{x} + \dots + a_1(t, x) \pdv{x} + a_0(t,x), \\
	M = \pdv[m]{x} + b_{m-1}(t, x) \pdv[m-1]{x} + \dots + b_1(t, x) \pdv{x} + b_0(t,x).
\end{align*}
We require the Lax equations to hold, then they are non-linear PDEs for coefficients $(a_0, \dots, a_{n-1},  b_0, \dots,b_{m-1})$. 

The linear $n$th-order PDE \eqref{math:3.3} 
\begin{equation}
	L \hat{f} = u \hat{f},
	\label{math:L-eigen}
\end{equation}
is equivalent to first-order matrix PDE
\begin{equation*}
	\pdv{F}{x}= U_L F,
\end{equation*}
with $n \times n$-matrix
\begin{equation*}
	U_L = \begin{pmatrix} 0 & 1 & 0 & \dots & 0 & 0 \\
								 0 & 0 & 1 & \dots & 0 & 0 \\
							 \vdots & \vdots & \vdots & \ddots & \vdots & \vdots\\ 
							 0 & 0 & 0 & 0 & 0 & 1 \\
							 u-a_0 & -a_1 & -a_2 & \dots & -a_{n-2} & -a_{n-1}
						 \end{pmatrix},
\end{equation*}
and $F = (f_0, f_1, \dots, f_{n-1})^T$ where $f_k = \pdv[k]{\hat{f}}{x}$.

Consider the second equation in \eqref{math:3.3}. Differentiating this equation with respect to $x$ and using \eqref{math:L-eigen} to express $\partial_x^{n} \hat{f}$ in terms of $u$ and lower order derivatives, and repeating this $(n-1)$ times gives an action of $M$ on components of the vector $F$
\begin{equation*}
	\pdv{F}{t} = V_m F.
\end{equation*}
Zero curvature compatibility conditions are now
\begin{equation}
	\partial_t U_L - \partial_x V_m + \comm{U_L}{V_m} = 0,
\end{equation}
if $(L, M)$ satisfy Lax equations.

\begin{example}
(KdV) 

KdV Lax pairs are 
\begin{equation*}
	L = - \pdv[2]{x} + \phi(t, x),\quad M = 4 \pdv[3]{x} - 3 \left(\phi \pdv{x} + \pdv{x} \phi\right).
\end{equation*}
Set $f_0 = \hat{f} (t, x, u)$ and $f_1 = \partial_x \hat{f}(t, x, u)$. Now the eigenvalue equation \eqref{math:L-eigen} gives
\begin{equation}
	(f_0)_x = f_1, \quad (f_1)_x = (\phi - u) f_0.
	\label{math:f_0_1-eq}
\end{equation}
The second equation $\partial_t \hat{f} + M \hat{f}$ gives
\begin{equation*}
	(f_0)_t = - 4 (f_0)_{x x x} + 6 \phi f_1 + 3 \phi_x f_0 = - \phi_x f_0 + (2 \phi + 4 u) f_1,
\end{equation*}
where the equation \eqref{math:L-eigen} has been used in the last step. Taking $\partial_x$ and using the equation \eqref{math:f_0_1-eq}
\begin{equation*}
	(f_1)_t = \left[ (2\phi + 4 u) (\phi - u) - \phi_{x x} \right]  f_0 + \phi_x f_1.
\end{equation*}

Collect equations in matrix form 
\begin{equation*}
	\partial_x F = U_L F, \quad 
	\partial_t F = V_m F
\end{equation*}
where $F = (f_0, f_1)^T$
\begin{equation*}
	U_L = \begin{pmatrix} 0 & 1 \\ \phi - u & 0 \end{pmatrix},\quad
	V_m = \begin{pmatrix} - \phi_x & 2\phi + 4 u \\ 2 \phi^2 - \phi_{x x} + 2\phi u - 4 u^2 & \phi_x \end{pmatrix}
\end{equation*}
This is the zero-curvature representation of KdV. \footnote{(Sign difference might come from the minus sign in the Schrödinger operator.)}
\end{example}

There is a gauge freedom in the zero-curvature representation $(U, V)$ and Lax pair $(L, M)$ (they are in general not unique). Consider an invertible matrix $g=g(t,x)$, then equation \eqref{math:lax_eq} and \eqref{math:zero-curv} are invariant under the transformation
\begin{align*}
	&U \rightarrow g U g^{-1} + \dv{g}{x} g^{-1},\quad  V \rightarrow g V g^{-1} + \dv{g}{t} g^{-1}, \\
	&L \rightarrow g L g^{-1},\quad  M \rightarrow g M g^{-1} + \dv{g}{t} g^{-1}
	\label{eq:}
\end{align*}

\chapter{Poisson Structures and Classical Yang-Baxter equation}
Goal is to find appropriate formulation of classically integrable systems with matrix Lax pair.

\section{Lax Pairs and Classical $r$-matrix}
Consider Lax pair of matrices satisfies
\begin{equation*}
	\dv{t} L = \comm{M}{L},
\end{equation*}
it implies one can do the transformation
\begin{equation*}
	L(t) = g(t) L(0) g^{-1}(t),
\end{equation*}
with 
\begin{equation*}
	M = \dv{g}{t} g^{-1}.
\end{equation*}
If $I(L)$ is a function of $L$ invariant under conjugation, $L \rightarrow g L g^{-1}$, then $I(L(t))$ is a constant of motion.

Suppose $L$ is diagnolizable $L = A \Lambda A^{-1}$ with 
\begin{equation*}
	\Lambda = \begin{pmatrix} u_1 & \dots & 0 \\
		\vdots & \ddots & \vdots \\
	0 & \dots & u_N \end{pmatrix}.
\end{equation*}

Define $I_n = \tr(L^{n})$  with $\dot{I}_n = \tr(L^{n-1} \comm{M}{L}) = 0$. One can extract eigenvalues $u_k$ from $I_n = \tr(\Lambda^n) = u_1^n + \dots + u_N^n$, so the eigenvalues $u_k$ are conserved. Question now is: are eigenvalues in involution?

\noindent\fbox{%
\begin{minipage}{40em}
% \parbox{\textwidth}{%
\paragraph{Notation}
We denote the canonical basis for $N \times N$ matrices as $(E_{\alpha \beta})_{\gamma \delta} = \delta_{\alpha \gamma} \delta_{\beta \delta}$ such that 
\begin{equation*}
	L = \sum_{\alpha \beta} L_{\alpha \beta} E_{\alpha \beta},
\end{equation*}
with $L_{\alpha \beta}$ being functions on phase space. Let  
\begin{equation*}
	L_1 := L \otimes \id = \sum_{\alpha \beta} L_{\alpha \beta} (E_{\alpha \beta} \otimes \id),
\end{equation*}
with $\id$ the $N \times  N $ identity matrix and $\otimes$ the tensor product. Then
\begin{equation*}
	L_2 := \id \otimes L_{\alpha \beta} = \sum_{\alpha \beta} L_{\alpha \beta} (\id \otimes E_{\alpha \beta}).
\end{equation*}

For a matrix $T$ living in the tensor product of two copies of $N \times  N$ matrices, set 
\begin{equation*}
	T = T_{12} = \sum_{\alpha \beta \gamma \delta} T_{\alpha \beta, \gamma \delta} E_{\alpha \beta} \otimes E_{\gamma \delta}, \quad 
	T_{21} = \sum_{\alpha \beta \gamma \delta} T_{\alpha \beta, \gamma \delta}  E_{\gamma \delta} \otimes E_{\alpha \beta}.
\end{equation*}
We may write $T_{21} = P_{12} T_{12} P_{12}^{-1}$ with $P_{12} = P_{12}^{-1}$ being the permutation operator of specs $1$ and $2$. Most generally, $L_k$ acts on space $k$ and $T_{jk}$ acts on spaces $j$ and $k$, e.g. $L_3 = \id \otimes \id \otimes L \otimes \dots$.

Denote by $\Tr_1$ the partial trace over space $j$ in the tensor product, e.g.
\begin{equation*}
	\Tr_1 T_{12} = \sum_{\alpha \beta \gamma \delta} T_{\alpha \beta, \gamma \delta} \tr(E_{\alpha \beta}) E_{\gamma \delta}.
\end{equation*}

Define $\pb{L_1}{L_2}$ as the matrix of Poisson brackets between the elements of $L$
\begin{equation*}
	\pb{L_1}{L_2} = \sum_{\alpha \beta \gamma \delta } \pb{L_{\alpha \beta}}{L_{\gamma \delta}} E_{\alpha \beta} \otimes E_{\gamma \delta}.
\end{equation*}
\end{minipage} 
}
\todo{What do the difference tensor spaces mean physically?}

%%%%%%%%%%%%%%%%%%%%%%%%%%%% lecture 7

For an integrable system the Poisson brackets between the elements of the Lax matrix can be written in a special form

% TODO: number proposition with section
\begin{prop}
	The involution property of the eigenvalues of $L$ is equivalent to the existence of a certain function $r_{12}$ on the phase space such that 
	\begin{equation}
		\pb{L_1}{L_2} = \comm{r_{12}}{L_1} - \comm{r_{21}}{L_2}
		\label{math:4.1}
	\end{equation}
\end{prop}

\begin{proof}
	Forward direction: assume that eigenvalues of $L$ Poisson commute, i.e. $\pb{u_j}{u_k}=0$. Consider
	\begin{align*}
		\pb{L_1}{L_2} = \pb{A_1 \Lambda_1 A_1^{-1}}{A_2 \Lambda_2 A_2^{-1}}
	\end{align*}
	and the right-hand side has $8$ terms after expansion. 
	There are four terms involve $\pb{A_1}{A_2}$ and can be written as
	\begin{align*}
		\comm{\comm{K_{12}}{L_2}}{L_1} = \frac{1}{2} \comm{\comm{K_{12}}{L_2}}{L_1} - \frac{1}{2} \comm{\comm{K_{21}}{L_1}}{L_2},
	\end{align*}
	with $K_{12} = \pb{A_1}{A_2} A_1^{-1} A_2^{-1}$. Jacobi identity and $K_{12} =- K_{21}$ have been used. \todo{There seems to be some special properties of $A$ matrices used.} There are other four terms with $\pb{\Lambda_1}{A_2}$ and $\pb{A_1}{\Lambda_2}$ and can be written as 
	\begin{equation*}
		\comm{q_{12}}{L_1} - \comm{q_{21}}{L_2},
	\end{equation*}
	with $q_{12} = A_2 \pb{A_1}{\Lambda_2} A_1^{-1} A_2^{-1}$.

	One finds 
	\begin{equation*}
	\pb{L_1}{L_2} = A_1 A_2 \pb{\Lambda_1}{\Lambda_2} A_1^{-1} A_2^{-1} + \comm{r_{12}}{L_1} - \comm{r_{21}}{L_2},
	\end{equation*}
	with $r_{12} = q_{12} + \frac{1}{2} \comm{K_{12}}{L_2}$. If the eigenvalues are in involution, equation \eqref{math:4.1} is valid.

	Backward direction: suppose we have $\pb{L_1}{L_2} = \comm{r_{12}}{L_1} - \comm{r_{21}}{L_2}$. Then
	\begin{equation}
		\pb{L_1^n}{L_2^m} = \comm{a_{12}^{n, m}}{L_1} + \comm{b_{12}^{n, m}}{L_2}
		\label{math:4.2}
	\end{equation}
	with 
	\begin{equation*}
		a_{12}^{n, m} = \sum_{p=0}^{n-1} \sum_{q=0}^{m-1} L_1^{n-p-1} L_2^{m-q-1} r_{12} L_1^p L_2^q,
	\end{equation*}
	and 
	\begin{equation*}
		b_{12}^{n, m}\sum_{p=0}^{n-1} \sum_{q=0}^{m-1} L_1^{n-p-1} L_2^{m-q-1} r_{21} L_1^p L_2^q.
	\end{equation*}

	Taking the race of \eqref{math:4.2} and using that $\tr(\comm{\cdot}{\cdot})=0$, one finds that the functions $\tr(L^n) = \tr(\Lambda^n) = u_1^n + \dots + u_N^n$ are in involution. Thus, eigenvalues $u_k$ of $L$ are in involution.
\end{proof}

Question now is: which restrictions on $r$-matrix follow from the Jacobi identity for the Poisson bracket \eqref{math:4.1}
\begin{equation}
	\comm{L_1}{\comm{r_{12}}{r_{13}} + \comm{r_{12}}{r_{23}} + \comm{r_{32}}{r_{13}} + \pb{L_2}{r_{13}} - \pb{L_3}{r_{12}}  } 
	% + (1\rightarrow2, 2 \rightarrow 3) + (1 \rightarrow 3, 2 \rightarrow 1)
	+ (\text{cycl. perm.})
	= 0.
	\label{math:r-matrix-jacobi}
\end{equation}
In general, it is not easy to capture. Assume that $r$ is a constant $r$-matrix, i.e. indepdent of the dynamical variables (Poisson brackets vanish in \eqref{math:r-matrix-jacobi}). Then a sufficient criterion for \eqref{math:r-matrix-jacobi} to hold is 
\begin{equation}
	\comm{r_{12}}{r_{13}} + \comm{r_{12}}{r_{23}} + \comm{r_{32}}{r_{13}} = 0
\end{equation}

For $r_{12} = - r_{21}$, this equation is the classical Yang-Baxter equation (CYBE)
\begin{equation}
\boxed{%
	\comm{r_{12}}{r_{13}} + \comm{r_{12}}{r_{23}} + \comm{r_{13}}{r_{23}} = 0
}
	\label{math:4.3}
\end{equation}

\begin{example} (harmonic oscillator)
Consider the dynamical $r$-matrix \todo{$r$-matrix, similar to the Lax pair, is not in general unique. One needs to find one satisfy the properties.}
\begin{equation*}
	r_{12} = - \frac{\omega}{4H} \begin{pmatrix} 0 & 1 \\ -1 & 0\end{pmatrix} \otimes L = - \frac{i\omega}{4H} \sigma_2 \otimes L,
\end{equation*}
with $H = \frac{1}{2} (p^2 + \omega^2 q^2)$ and $\pb{q}{p} = 1$. The Lax pair for the harmonic oscillator is (with $m=1$)
\begin{equation*}
	L = \begin{pmatrix} p & \omega q \\ \omega q & -p \end{pmatrix} = p\sigma_3 + \omega q_1 \sigma_1
\end{equation*}
and 
\begin{equation*}
	M = \begin{pmatrix} 0 & - \frac{\omega}{2} \\ \frac{\omega}{2} & 0\end{pmatrix} = -\frac{i\omega}{2} \sigma_2
\end{equation*}
The eigenvalues of $L$ are $\pm \sqrt{H}$.
\end{example}

\begin{example} (Lax pairs with spectral parameters)

Often a Lax pair depends on a spectral parameter $u$, such that 
\begin{equation}
	\dv{t} L(u) = \comm{M(u)}{L(u)}
	\label{math:4.4}
\end{equation}
as before, the $H_n(u) = \tr(L(u)^n)$ are integrals of motion for any $u$, then $H_n(u) = \sum_{k} u^k H_{n, k}$ and it generates integrals of motion $H_{n, k}$.  Also, the above $r$-matrix and CYBE become spectral parameter dependent.
\end{example}

\begin{prop}
	Suppose that \eqref{math:4.1} holds. If we take $H_n = \tr(L^n)$ as Hamiltonians, then the equation of motion admit a Lax representation
	\begin{equation*}
		\dv{L}{t_n} := \pb{H_n}{L} \stackrel{!}{=} \comm{M_n}{L}
	\end{equation*}
	with $M_n = - n \tr_1 (L_1^{n-1} r_{21})$. $t_n$ is the time generated by the Hamiltonian $H_n$,
\end{prop}

\begin{example}  (spectral parameter dependent Lax pair)

Remember the Lax pair from E$7$
\begin{equation*}
	L = \vec{S} \cdot \vec{\sigma},\quad M = - \frac{i}{2} (\Omega^{-1} \vec{S}) \cdot \vec{\sigma}
\end{equation*}
Modify $L$ to $L(u) = L + u H \cdot \id$. It still obeys Lax equation $\dot{L}(u) = \comm{M}{L(u)}$. Spectrum of $L(u)$ includes Hamiltonian: eigenvalues $\left\{ uH + l, uH-l \right\} $ where $l = |\vec{S}|$. 

Classical $r$-matrix is now
\begin{equation*}
	r_{12} = - \frac{i}{4} \vec{\sigma_1} \cdot \vec{\sigma_2} - \frac{i}{2} u (\Omega^{-1} \vec{S}) \cdot \vec{\sigma}_1.
\end{equation*}
(Index of $\vec{\sigma}$ denotes the space not the component.)
\end{example}

We have decomposed matrices as 
\begin{equation*}
	L = \sum_{\alpha \beta} L_{\alpha \beta} E_{\alpha \beta}
\end{equation*}
or 
\begin{equation*}
	r_{12} = \sum_{\alpha \beta \gamma \delta} r_{\alpha \beta, \gamma \delta} E_{\alpha \beta} \otimes E_{\gamma \delta}
\end{equation*}
Here $E_{\alpha \beta}$ is a canonical basis of Lie algebra $gl(N)$. Consider $r_{12}$ as an element $gl(N) \otimes gl(N)$.

Generalize this framework to a Lie algebra $g$ with basis of generators denoted by $t_a$ and $\comm{t_a}{t_b} = f_{abc} t_c$ with $f$ the structure constant. Then we consider $r_{12} \in g \otimes g$. Note that we do not distinguish upper and lower indices $a, b, c$ here.

\section{Classification and Algebraic Structure of Integrable $r$-matrices}
Goal is to understand how integrable $r$-matrices look like

\begin{theorem} (Belavin-Drinfeld I)
	Let $g$ be a finite dimensional simple Lie algebra, and $r = r(u_1 - u_2): \Co \rightarrow g \otimes g$ a solution of the (spectral-parameter depedent) classical YBE 
	\begin{equation}
		\comm{r_{12}(u_{12})}{r_{13} (u_{13})} + \comm{r_{12}(u_{12})}{r_{23}(u_{23})} + \comm{r_{13}(u_{13})}{r_{23}(u_{23})} = 0
		\label{math:4.5}
	\end{equation}
	with $u_{ij} = u_{i} - u_j$.

	Furthermore, assume one of the following three equivalent conditions holds
	\begin{enumerate}
		\item $r$ has at least one pole in the complex plane $u = u_1 - u_2$ and there is no Lie subalgebra $g' \subset g$ such that $r \in g' \otimes g'$for any $u$.
		\item $r(u)$ has a simple pole at the origin, with residual proportional to $C_{\otimes} = \sum_a t_a \otimes t_a$ with $\left\{ t_a \right\} $ being a basis of $g$ orthonormal with respect to a chosen non-degenerate  bilinear form.
		\item The determinant of the matrix $r^{ab}(u)$ obtained from 
			\begin{equation*}
				r(u) = \sum_{a, b} r^{ab}(u) t_a \otimes t_b
			\end{equation*}
			does not vanish identically.
	\end{enumerate}

	Under those assumptions $r_{12}(u) = - r_{21}(-u)$, where $r_{21}(u) = \mathcal{P} r_{12} (u) \mathcal{P} = \sum_{a, b} r^{ab} (u ) t_b \otimes t_a$ and $r(u)$ can be extended meromorphically to the entire $n$-plane. All the poles of $r(u)$ are simple and they form a lattice $\Gamma$. One has three possible equivalent classes of solutions
	\begin{enumerate}
		\item rational solution: $\Gamma = \left\{ 0 \right\} $.
		\item trigonometric: $\Gamma$ is a one-dimensional array
		\item elliptic: $\Gamma$ is a two-dimensional lattice
	\end{enumerate}
\end{theorem}

\begin{example} (classical $r$-matrices)
\begin{enumerate}
	\item $r(u) = \frac{1}{u} C_\otimes$ with $C_\otimes = \sum_a t_a \otimes t_a$.
	\item $r(u) = \frac{1}{\sinh(u)} \begin{pmatrix} (\frac{1}{2} + \frac{1}{2} \sigma_z) \cosh u & \sinh(i u) \sigma^{-} \\ \sinh(i u) \sigma^+ & (\frac{1}{2} - \frac{1}{2} \sigma_z) \cosh(u) \end{pmatrix}$.
	\item Belavin-Drinfeld showed that elliptic solutions only exist for $g = sl(N)$.
\end{enumerate}
\end{example}

%%%%%%%%%%%%%%%%%%%%%%%%%%%%%%%% lecture 8
The assumption to have an $r$-matrix of difference form is not too restrictive. 

\begin{theorem} (Belavin-Drinfeld II)
	Given the assumptions of B.-D. I but with $r(u_1, u_2)$	 not of difference form and with the classical YBE 
	\begin{equation}
	\comm{r_{12}(u_1, u_2)}{r_{13} (u_1, u_3)} + \comm{r_{12}(u_1, u_2)}{r_{23}(u_2, u_3)} + \comm{r_{13}(u_1, u_3)}{r_{23}(u_2, u_3)} = 0
	\end{equation}
	Assume that condition (ii) holds. Now (i)-(iii) are no longer equivalent. Assume that dual Coxeter number $c_2$ of our $g$ be non-zero, where $c_2$ is defined by $f_{abc} f_{bcd} = c_2 \delta_{ad}$. Then there exists a transformation which reduces $r$ the the difference form.
\end{theorem}

The two Belavin-Drinfeld theorems allow us to classify classical integrable structures as well as their possible quantizations.

\paragraph{Algebraic properties of the classical $r$-matrix}
Consider $r$-matrix as function of $u \in \Co$. 

\begin{example} (Yang's $r$-matrix)	
	The prototype of a rational solution to the classical cYBE is
	\begin{equation}
		r(u_1 - u_2) = c \frac{C_{\otimes}}{u_2 - u_1}
	\end{equation}
	with $C_\otimes = t_a \otimes t_a$ denoting the tensor Casimir.

	One can expand the $r$-matrix as 
	\begin{equation}
		\frac{r}{c} = \frac{t_a \otimes t_a}{u_2 - u_1} = \sum_{n \geq 0} t_a u_1^n \otimes t_a u_2^{-n-1} =: \sum_{n \geq 0} t_{a, n} \otimes t_{a, -n-1}
	\end{equation}
	where $\frac{u_2}{1-q} = u_2 \sum_n q^{n}$ has been used and we assume $|u_1 / u_2| < 1$.

	It seems natural to define $t_{a, n} = u^n t_a$ with 
	\begin{equation}
		\comm{t_{a, m}}{t_{b, n}} = f_{abc} t_{c, m+n}
		\label{math:4.7}
	\end{equation}
\end{example}

\begin{definition}
	\label{def:4.1}
	The algebra spanned by the $t_{a, n}$ is called the loop algebra $g[u, u^{-1}]$.	
\end{definition}

Using \eqref{math:4.7}, one can show that the cYBE is satisfied by $r = \sum_{n \geq 0} t_{a, n} \otimes t_{a, -n-1}$. It does not depend on representation. Thus, Yang's $r$-matrix lives in $g[u_1, u_1^{-1}] \otimes g[u_2, u_2^{-1}]$.

\begin{definition}
	\label{def:4.2}
	Given a Lie algebra $g$, the universal enveloping algebra (UEA) $U[g]$ is defined as the quotient of the tensor algebra $K \oplus g \oplus (g \otimes g) \oplus (g \otimes g \otimes g) \oplus \dots $ by the elements $b \otimes a - b \otimes a - \comm{a}{b}$ for $a, b \in g$. That is $t_a \otimes t_b - t_b \otimes t_a$ is identified with $f_{abc}t_c$.

	The UEA can be equipped with a Hopf algebra structure with 
	\begin{itemize}
	\item  coproduct
	\begin{equation}
		\Delta (a) = a \otimes \id + 1 \otimes a
	\end{equation}
	\item antipole
	\begin{equation}
		s(a) = 0
	\end{equation}
	\item co-unit
	\begin{equation}
		\epsilon(a) = 0
	\end{equation}
	for $a \in g$.
	\end{itemize}
\end{definition}

\begin{example} 
	The quadratic Casimir $t_a t_a$  or $t_a \otimes t_a$ is element of $U[g]$.
\end{example}

The above coproduct is called co-commutative since $\Delta^{\text{op}} := \mathcal{P} \Delta \mathcal{P} = \id \otimes a + a \otimes \id = \Delta$, where $\Delta^{\text{op}}$ is the opposite coproduct.

Quantization of this algebraic structure leads to the definition of a non-co-commutative coproduct with 
\begin{equation*}
	\Deltaop - \Delta \sim \hbar
\end{equation*}
Remember: $xp - px = 0$ but after quantization $\hat{x} \hat{p} - \hat{p} \hat{x} = i \hbar$. Quantum groups/algebra can be understood as deformations $U_\hbar [g]$ of universal enveloping algebras.

\paragraph{More algebraic notions}
One may introduce the concept of Lie algebra for which $r$-matrix is of central importance. A Lie bialgebra is called 
\begin{itemize}
	\item  co-boundary: if it has an $r$-matrix
	\item quasi-triangular: if $r$-matrix obeys the cYBE
	\item triangular: if $r$-matrix is anti-symmetric.
\end{itemize}

\chapter{Field Theories in $1+1$ Dimensions}

\section{Classical Field Theory}
Field is an object $\phi(t, \vec{x})$ deinfed at every point in spacetime. Dynamics of fields governed by Lagrangian density $\lag (\phi_A, \partial_\mu \phi_A)$ and the action
\begin{equation*}
	S = \int \dd{t} \int \dd[d-1]{x} \lag = \int \dd[D]{x} \lag
\end{equation*}
The Euler-Lagrange equations are
\begin{equation*}
	\partial_\mu \left( \pdv{\lag}{(\partial_\mu \phi_A)} \right)  - \pdv{\lag}{\phi_A} = 0
\end{equation*}

Every continuous symmetry of the Lagrangian gives rise a conserved current $j^{\mu}(x)$ such that the e.o.m. imply the conservation equation
\begin{equation}
	\partial_\mu j^{\mu} = 0,
	\label{math:5.1}
\end{equation}
or in other words
\begin{equation*}
	\pdv{j^{0}}{t} + \vec{\nabla} \cdot \vec{j} = 0.
\end{equation*}

A conserved current implies a conserved charge $J$
\begin{equation}
	J = \int_{\R^{D-1}} \dd[D-1]{x} j^0(t, \vec{x})
\end{equation}
since 
\begin{align*}
	\dv{J}{t} = \int \dd[D-1]{x} \pdv{j^0}	{t} = - \int \dd[D-1]{x} \vec{\nabla} \cdot \vec{j} = 0,
\end{align*}
where we assume $\lim_{|\vec{x}|\rightarrow 0} \vec{j} = 0$.

We are interested in $D=1 + 1$ dimensions where $\mu = 0, 1$ and metric $\eta_{\mu\nu} = \begin{pmatrix} +1 & 0 \\ 0 & -1 \end{pmatrix}$, $\epsilon_{\mu\nu} = \begin{pmatrix} 0 & 1 \\ -1 & 0\end{pmatrix}$ and $\epsilon^{\mu\nu} = \begin{pmatrix} 0 & -1 \\ 1 & 0\end{pmatrix}$. Thus, $j_0 = j^0$, $j_1 = - j^1$ and $j_\mu = \eta_{\mu\nu} j^\nu$.

\paragraph{Nonlocal symmetries}
Suppose we have a conserved current $j^\mu$ that obeys the zero-curvature condition
	\begin{equation}
		\partial_0 j_1 - \partial_1 j_0 + \comm{j_0}{j_1} = 0
		\label{math:5.2}
	\end{equation}
	(we have $U = -j_1$ and $V = -j_0$.) We say that this current is flat. Suppose the current is Lie algebra valued, i.e. $j^\mu \in g$ with $j_\mu = j_{\mu a} t_a$ with $\comm{t_a}{t_b} = f_{abc} t_c$ and zero curvature condition
	\begin{equation}
		\partial_0 j_{1a} + \partial_1 j_{0a} + f_{abc} j_{0b} j_{1c} = 0
	\end{equation}

Define a bilocal current
\begin{equation*}
	\hat{j}_a^\mu (t, x) = \epsilon^{\mu\nu} j_{\nu a} (t, x) \frac{1}{2} f_{abc} j_b^\mu (t, x) \int_{-\infty}^x \dd{y} j_c^0 (t, y)
\end{equation*}
Using flatness and conservation of $j^\mu$, we find that $\hat{j}$ is conserved $\partial_\mu \hat{j}^\mu = 0$ and the conserved charge 
\begin{equation*}
	\hat{J}_a = \int_{-\infty}^{\infty} \hat{j}_a^0 (t,x)	
\end{equation*}
with $\dv{\hat{J}}=0$ if $\lim_{|x|\rightarrow 0} j(t, x) =0$.

\subsection{Monodromy- and Transfer-matrix}
Call classical field theory integrable, if two (spectral-parameter-dependent) matrices $U$ and $V$ exist such that the (Euler Lagrange) equations of motion can be written as a zero-curvature condition
\begin{equation}
	\partial_t U - \partial_x V + \comm{U}{V} = 0
	\label{math:chap5-zero-curv}
\end{equation}

Alternatively, write this as 
\begin{equation*}
	\comm{\D_\mu (u)}{\D_\nu (v)} = 0
\end{equation*}
with $D_\mu (u) = \partial_\mu - L_\mu (u)$ and $L_0 = V$ and $L_1 = U$. \eqref{math:chap5-zero-curv} is compatibility conditions for the auxiliary linear problem $\D_\mu \Phi = 0$.

Consider transport matrix $T(t, x_0, x)$ transporting solutions along interval $[x_0, x]$
\begin{equation*}
	\Phi(t, x) = T(t, x_0, x) \Phi(t, x_0)
\end{equation*}
with $\D_1 T = 0$ and $T(t, x_0, x_0) = \id$. The solution is 
\begin{equation}
	T(t, x_0, x) = P \exp[\int_{x_0}^x \dd{x'} U(t, x')]
	\label{math:5.3}
\end{equation}
with $P \exp = \revvec{\exp}$ the path-ordering with greater $x$ to the left. More generally, $T = \revvec{\exp}\left[\int_{\gamma} (U \dd{x} + V \dd{t}) \right]$ (zero curvature means path indepdendence)

Consider time-derivative of $T$: Understand $\revvec{\exp} [\int_{x_0}^{x} U(x') \dd{x'}] \sim (1+\delta_x U(x_n)) \dots (1+ \delta_x U(x_0))$ with $x_0 < x_1 < \dots < x_n = x$ such that $x_{i+1} - x_i = \delta_x \rightarrow 0$.

\begin{align*}
	\partial_t T &= \int^x_{x_0} \dd{x'} \exp[\int_{x'}^x U\dd{x''} ] \dot{U}(x') \exp[\int_{x_0}^{x'} U \dd{x''}] \\
					 &= \int_{x_0}^x \dd{x'} \exp[\int_{x'}^x U \dd{x''} ] \left( \pdv{V}{x'} - \comm{U}{V} \right)  \exp[\int_{x_0}^{x'} U\dd{x''}]  \\
					 &= \int_{x_0}^{x} \dd{x'} \partial_{x'} \left( \exp[\int_{x'}^{x} U \dd{x''}] V(x') \exp[\int_{x_0}^{x'} U \dd{x''}] \right) \\
					 &= V(x) T - T V(x_0)
\end{align*}
where all $\exp$'s are understood as path-ordered.

Set 
\begin{equation}
	T(u) = T(t, S_-, S_+, u)
\end{equation}
where $S_\pm$ are boundaries of space. This is the \textit{monodromy matrix}.

Distinguish boundary conditions
\begin{enumerate}
	\item Case $1$: infinite line $S_\pm = \pm \infty$ (e.g. standard field theory). Assume $V(\pm \infty) = 0$, hence $\partial_t T(u) = 0$.
	\item Case $2$: periodic boundaries: $S_- \simeq S_+$. Then, $\partial_t T = \comm{V(t, u)}{T}$ which is the Lax equation.
\end{enumerate}

%%%%%%%%%%%%%%%%%%%%%%%%%%%%%%%%%% lecture 9

\begin{example} (Case $1$) 
	Field theory with conserved and flat current $j^\mu$. Define $\D_\mu(u) =  \partial_\mu - L_\mu (u)$ with $L_\mu = \frac{1}{u^2 - 1} \left[ j_\mu + u \epsilon_{\mu\nu} j^\nu \right] $.

	Then we have $\comm{\D_\mu (u)}{\D_\nu (u)} = 0$ and $j^\mu ( t, \pm \infty) = 0$, thus $V(t, \pm \infty) = L_0 (t, \pm \infty) = 0$.

	After expanding around $u=\infty$, the monodromy matrix is 
	\begin{equation}
		T(u) = 1 - \frac{1}{u} \int_{-\infty}^{\infty} \dd{x} j_0 (x) + \frac{1}{u^2} \left[ \int_{-\infty}^{\infty} \dd{x} j_1 (x) + \int_{\infty}^{\infty} \dd{x} \int_{-\infty}^{\infty} \dd{y} j_0(t, x) j_0 (t, y) \right]  + \order{\frac{1}{u^3}}
	\end{equation}
	The second integral is the local charge $J$ and the quantity is brackets is the bilocal charge, $\simeq \hat{J} - J^2$. Higher order terms are higher non-local charges.
\end{example}


\begin{example} (Case $2$: periodic boundaries)
	Lax equation $\partial_t T = \comm{V}{T}$	. Define the transfer matrix as $\mathcal{t}$ as 
	\begin{equation}
		\mathcal{t}(u) = \tr(T(u))
		\label{math:5.5}
	\end{equation}
	and thus $\partial_t \mathcal{t} (u) = 0, \forall u$.

	Expansion yields family of $u$-independent conserved charges
	\begin{equation}
		\mathcal{t}(u) = \sum_{n \leq 0} u^n Q_n
	\end{equation}
	with $\partial_t Q_n = 0$.

	One can loo at Poisson brackets and Lax matrix. Suppose that the canonical Poisson brackets imposed on the fields imply the following \textit{ultralocal} brackets for a matrix $L$
	\begin{equation}
		\pb{L_1(t, x, u)}{L_2 (t, y, u')} = \comm{r_{12}(u-u')}{L_1(t, x, u) + L_2 (t, y, u')} \delta(x-y)
		\label{math:5.6}
	\end{equation}
	``Ultralocal'' means there is only $\delta$-function, but no $\delta '$.

	Furthermore, we assume that $r_{12}(u - u')$ does not depedent on the fields and satisfies 
	\begin{equation}
		r_{12}(u - u') = - r_{21} (u' - u)
	\end{equation}
\end{example}

\begin{theorem} (Sklyanin Exchange Relations)
	Given \eqref{math:5.6}, the Poisson brackets of the monodromy $T(u) = \exp(L(t, x, u) \dd{x})$	satisfies 
	\begin{equation}
		\pb{T_1(u)}{T_2(u')} =  \comm{r_{12}(u - u')}{T_1(u) T_2(u')}
		\label{math:5.7}
	\end{equation}
	One can conclude that the conserved charges generated by the transfer matrix $\mathcal{t}(u) = \tr(T(u))$ are in involution: apply $\tr_1 \otimes \tr_2$ to \eqref{math:5.7}
	\begin{equation*}
		\pb{t(u)}{t(u')} = 0
	\end{equation*}
	by cyclicity.
\end{theorem}

The functions $L$ and $T$ furnish the most convenient language to capture field theory integrability. Also quantization can be based on \eqref{math:5.7}.

\chapter{Quantum Yang-Baxter Equation}
\section{On the Definition of Quantum Integrability}
Remember classical integrability for finite-dimensional system could be defined by existence of $N$ algebraically independent integrals of motions $\pb{I_i}{I_j} = 0, \forall i, j$ and $\pb{H}{I_i} = 0, \forall i$.

Ideally, one would quantize the system by
\begin{equation*}
	q \rightarrow \hat{q}, p \rightarrow \hat{p}, H \rightarrow \hat{H}, \pb{}{} \rightarrow \comm{}{}
\end{equation*}
The naive definition of quantum integrability would be the existence of $N$ independent operators $\hat{I}_1, \dots, \hat{I}_N$ that commute with $\hat{H}$ and among each other.

What should independent mean here? Assume the spectrum is non-degenerate (to avoid some subtlety) and $\comm{\hat{H}}{\hat{I}} = 0$, then one can find the simutaneous eigenstates
\begin{equation*}
	\hat{H} \ket{\psi_j} = E_j \ket{\psi_j}, \;
	\hat{I} \ket{\psi_j} = a_j \ket{\psi_j}, \;
	j = 1, \dots, N.
\end{equation*}
Hence, one may write
\begin{equation*}
	\hat{I} = \sum_{j=1}^N a_j \ket{\psi_j} \bra{\psi_j}	.
\end{equation*}

Then any $\hat{I}$ can be written as a polynomial in $\hat{H}$
\begin{equation*}
	\hat{I} = \sum_{j=1}^N a_j \prod_{k=1, k\neq j }^N \frac{\hat{H} - E_k}{E_j - E_k} = \sum_{j=1}^N \hat{H}^{k-1} \sum_{j=1}^N m_{kj} a_j,
\end{equation*}
where the $m_{kj}$ are functions of eigenvalues $E_l$ only.

Thus, no two commuting operators are algebraically independent, which at most $N$ commuting operators  which are linearly independent. However, linear independent are also $\left\{ \hat{H}, \hat{H}^2, \hat{H}^3, \dots \right\} $. The naive definition of quantum integrability is \textit{not good}.

\section{Factorized Scattering}
Scattering processes are essential to understand the world (e.g. LHC). Consider relativistic massive $(1+1)$-dimensional model. One space dimension implies ordering of particles is well-defined. Translate particle momentum $p^\mu_k (\mu = 0, 1)$ into rapidity
\begin{equation}
	p_k^0 = m \cosh(u_k),\;
	p_k^1 = m \sinh(u_k).
\end{equation}
which ensures the particle is on-shell $p^2 = (p^0)^2 - (p^1)^2 = m^2$.

Alternatively, one can use lightcone momenta
\begin{equation*}
	p^+ = p^0 + p^1 = m e^u, \;
	p^- = p^0 - p^1 = m e^{-u}.
\end{equation*}
which transform under Lorentz boosts $B_\alpha: u \rightarrow u + \alpha$
\begin{equation*}
	p^+ \rightarrow p^+ e^\alpha, p^- \rightarrow p^- e^{-\alpha}.
\end{equation*}

Tensors of the Lorentz group in $1+1$-dimensions are labeled by their Lorentz spin $s$ according to
\begin{equation*}
	B_\alpha: Q_s \rightarrow e^{s\alpha} Q_s ,
	\label{math:6.1}
\end{equation*}
Hence, $p^\pm$ have spin $s=\pm 1$.

Suppose $Q_s$ is local conserved quantity of spin $s > 0$ ($s<0$ from parity) in a scattering process of $n$ particles of type $A_i, i = 1, \dots n$ with masses $m_i$. $Q_s$ acts on one-particle states as  ($p_i^s = p_i^{+ s}$)
\begin{equation*}
	Q_s \ket{A_i(u)} \sim p^s_i \ket{A_i(u)}
\end{equation*}
Action on multi-particle states (due to locality of $Q_s$)
\begin{equation}
	Q_s \ket{A_1(u_1) \dots A_n (u_n)} \sim \sum_{k=1}^n p^s_k \ket{A_1 (u_1) \dots A_n (u_n)}
	\label{math:6.2}
\end{equation}

In a scattering process, we have (here $p=p^+, \bar{p} = p^-$)
\begin{equation}
	\sum_{i \in \left\{ \text{in} \right\} } p^s_i = \sum_{f \in \left\{ \text{out} \right\} } p_f^s \stackrel{\text{parity}}{\rightarrow} 
	\sum_{i \in \left\{ \text{in} \right\} } \bar{p}^s_i = \sum_{f \in \left\{ \text{out} \right\} } \bar{p}_f^s
	\label{math:6.3}
\end{equation}
For $s=1$ this is energy and momentum conservation. 

Consider integrable system with infinitely many conserved charges $Q_s$ of different spin $s$. By \eqref{math:6.3}
\begin{equation}
	\left\{ p_i^\mu | i \in \text{in} \right\}  = \left\{ p_f^\mu | f \in \text{out} \right\} 
\end{equation}
Thus there is no particle production or annihilation. Individual momenta are conserved.

\paragraph{Factorization}
Relativistic invariance means two-particle scattering matrix may only depend on
\begin{equation*}
	p^\mu_i p_j^\nu \eta_{\mu\nu} = m_i m_j \cosh(u_i - u_j).
\end{equation*}
From now on we write $u_{ij} := u_i - u_j$.

Since all momenta are conserved, the most general two-particle $S$-matrix is
\begin{equation}
	\ket{A_i (u_1) A_j (u_2)}_{\text{in}} = \sum_{k, l} S^{kl}_{ij} (u_{12}) \ket{A_k(u_2) A_l(u_1)}_{\text{out}}
\end{equation}

\begin{prop}
The $n$-particle $S$-matrix in a $2d$ integrable theory can always be written as the product of $ \begin{pmatrix} n \\ 2 \end{pmatrix}$ two-particle $S$-matrices.
\end{prop}

\begin{proof} (schematically)
Choose initial states with $n$ particles and $u_1 > u_2 > \dots > u_n$, $x_1 < x_2 < \dots < x_n$. After $n(n-1)/2$ pair collisions, the particles reach the infinite future in inverse ordering. Scattering described by 
\begin{equation*}
	S \ket{A_{i1} (u_1) \dots A_{in} (u_n)}_{\text{in}} =\sum_{j_1, \dots, j_n} S_{i1, \dots, in}^{j1, \dots, jn} \ket{A_{j1}(u_n) \dots A_{jn}(u_1)}_{\text{out}}.
\end{equation*}
Integrability: assume infinitely many local commuting and conserved operators $Q_s$
\begin{equation*}
	Q_s \ket{A_{i1} (u_1) \dots A_{in} (u_n)} =(q_s(u_1) + \dots + q_n(u_s)) \ket{A_{i1}(u_1) \dots A_{in}(u_n)},
\end{equation*}
with $q_s(u_1) \sim (p_1^+)^s$. Commuting charges means these can be simultaneously diagonalized. 

Consider particles as localized wave-packets:
\begin{equation*}
	\psi(x) \sim \int_{-\infty}^{\infty} \dd{p} e^{-a^2(p-p_1)^2} e^{i p (x - x_1)}.
\end{equation*}
An operator acting on $\psi$ gives momentum dependent phase factor 
\begin{equation*}
	\rightarrow \tilde{\psi} = \int_{-\infty}^{\infty} \dd{p} e^{-a^2(p-p_1)^2} e^{ip(x-x_1)} e^{-i\phi(p)},
\end{equation*}

Major contribution to integral close to $p_1$. Expand $\phi(p)$ and find modified values ($e^{-i\phi(p)} \simeq e^{-i\phi(p_1)} \cdot e^{-ip \phi'(p_1)}$)
\begin{equation*}
	\tilde{p}_1 = p_1, \tilde{x}_1 = x_1 + \phi'(p_1).
\end{equation*}
Position of particle $k$ is shifted by $\phi'(k)$.

Assume $Q_s \sim P^s$ and act with $e^{i\alpha Q_s}$, then $\phi_s(p) = \alpha p^s$. The particle with momentum $p_k$ is shifted by $s \alpha p_k^{s-1}$. ($s=1$: momentum operator $P$ shifts by constant.)
\end{proof}

\printbibliography
\end{document}
