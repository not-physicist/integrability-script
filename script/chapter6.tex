\chapter{Quantum Yang-Baxter Equation}
\section{On the Definition of Quantum Integrability}
Remember classical integrability for finite-dimensional system could be defined by existence of $N$ algebraically independent integrals of motions $\pb{I_i}{I_j} = 0, \forall i, j$ and $\pb{H}{I_i} = 0, \forall i$.

Ideally, one would quantize the system by
\begin{equation*}
	q \rightarrow \hat{q}, p \rightarrow \hat{p}, H \rightarrow \hat{H}, \pb{}{} \rightarrow \comm{}{}
\end{equation*}
The naive definition of quantum integrability would be the existence of $N$ independent operators $\hat{I}_1, \dots, \hat{I}_N$ that commute with $\hat{H}$ and among each other.

What should independent mean here? Assume the spectrum is non-degenerate (to avoid some subtlety) and $\comm{\hat{H}}{\hat{I}} = 0$, then one can find the simutaneous eigenstates
\begin{equation*}
	\hat{H} \ket{\psi_j} = E_j \ket{\psi_j}, \;
	\hat{I} \ket{\psi_j} = a_j \ket{\psi_j}, \;
	j = 1, \dots, N.
\end{equation*}
Hence, one may write
\begin{equation*}
	\hat{I} = \sum_{j=1}^N a_j \ket{\psi_j} \bra{\psi_j}	.
\end{equation*}

Then any $\hat{I}$ can be written as a polynomial in $\hat{H}$
\begin{equation*}
	\hat{I} = \sum_{j=1}^N a_j \prod_{k=1, k\neq j }^N \frac{\hat{H} - E_k}{E_j - E_k} = \sum_{j=1}^N \hat{H}^{k-1} \sum_{j=1}^N m_{kj} a_j,
\end{equation*}
where the $m_{kj}$ are functions of eigenvalues $E_l$ only.

Thus, no two commuting operators are algebraically independent, which at most $N$ commuting operators  which are linearly independent. However, linear independent are also $\left\{ \hat{H}, \hat{H}^2, \hat{H}^3, \dots \right\} $. The naive definition of quantum integrability is \textit{not good}.

\section{Factorized Scattering}
Scattering processes are essential to understand the world (e.g. LHC). Consider relativistic massive $(1+1)$-dimensional model. One space dimension implies ordering of particles is well-defined. Translate particle momentum $p^\mu_k (\mu = 0, 1)$ into rapidity
\begin{equation}
	p_k^0 = m \cosh(u_k),\;
	p_k^1 = m \sinh(u_k).
\end{equation}
which ensures the particle is on-shell $p^2 = (p^0)^2 - (p^1)^2 = m^2$.

Alternatively, one can use lightcone momenta
\begin{equation*}
	p^+ = p^0 + p^1 = m e^u, \;
	p^- = p^0 - p^1 = m e^{-u}.
\end{equation*}
which transform under Lorentz boosts $B_\alpha: u \rightarrow u + \alpha$
\begin{equation*}
	p^+ \rightarrow p^+ e^\alpha, p^- \rightarrow p^- e^{-\alpha}.
\end{equation*}

Tensors of the Lorentz group in $1+1$-dimensions are labeled by their Lorentz spin $s$ according to
\begin{equation*}
	B_\alpha: Q_s \rightarrow e^{s\alpha} Q_s ,
	\label{math:6.1}
\end{equation*}
Hence, $p^\pm$ have spin $s=\pm 1$.

Suppose $Q_s$ is local conserved quantity of spin $s > 0$ ($s<0$ from parity) in a scattering process of $n$ particles of type $A_i, i = 1, \dots n$ with masses $m_i$. $Q_s$ acts on one-particle states as  ($p_i^s = p_i^{+ s}$)
\begin{equation*}
	Q_s \ket{A_i(u)} \sim p^s_i \ket{A_i(u)}
\end{equation*}
Action on multi-particle states (due to locality of $Q_s$)
\begin{equation}
	Q_s \ket{A_1(u_1) \dots A_n (u_n)} \sim \sum_{k=1}^n p^s_k \ket{A_1 (u_1) \dots A_n (u_n)}
	\label{math:6.2}
\end{equation}

In a scattering process, we have (here $p=p^+, \bar{p} = p^-$)
\begin{equation}
	\sum_{i \in \left\{ \text{in} \right\} } p^s_i = \sum_{f \in \left\{ \text{out} \right\} } p_f^s \stackrel{\text{parity}}{\rightarrow} 
	\sum_{i \in \left\{ \text{in} \right\} } \bar{p}^s_i = \sum_{f \in \left\{ \text{out} \right\} } \bar{p}_f^s
	\label{math:6.3}
\end{equation}
For $s=1$ this is energy and momentum conservation. 

Consider integrable system with infinitely many conserved charges $Q_s$ of different spin $s$. By \eqref{math:6.3}
\begin{equation}
	\left\{ p_i^\mu | i \in \text{in} \right\}  = \left\{ p_f^\mu | f \in \text{out} \right\} 
\end{equation}
Thus there is no particle production or annihilation. Individual momenta are conserved.

\paragraph{Factorization}
Relativistic invariance means two-particle scattering matrix may only depend on
\begin{equation*}
	p^\mu_i p_j^\nu \eta_{\mu\nu} = m_i m_j \cosh(u_i - u_j).
\end{equation*}
From now on we write $u_{ij} := u_i - u_j$.

Since all momenta are conserved, the most general two-particle $S$-matrix is
\begin{equation}
	\ket{A_i (u_1) A_j (u_2)}_{\text{in}} = \sum_{k, l} S^{kl}_{ij} (u_{12}) \ket{A_k(u_2) A_l(u_1)}_{\text{out}}
\end{equation}

\begin{prop}
The $n$-particle $S$-matrix in a $2d$ integrable theory can always be written as the product of $ \begin{pmatrix} n \\ 2 \end{pmatrix}$ two-particle $S$-matrices.
\end{prop}

\begin{proof} (schematically)
Choose initial states with $n$ particles and $u_1 > u_2 > \dots > u_n$, $x_1 < x_2 < \dots < x_n$. After $n(n-1)/2$ pair collisions, the particles reach the infinite future in inverse ordering. Scattering described by 
\begin{equation*}
	S \ket{A_{i1} (u_1) \dots A_{in} (u_n)}_{\text{in}} =\sum_{j_1, \dots, j_n} S_{i1, \dots, in}^{j1, \dots, jn} \ket{A_{j1}(u_n) \dots A_{jn}(u_1)}_{\text{out}}.
\end{equation*}
Integrability: assume infinitely many local commuting and conserved operators $Q_s$
\begin{equation*}
	Q_s \ket{A_{i1} (u_1) \dots A_{in} (u_n)} =(q_s(u_1) + \dots + q_n(u_s)) \ket{A_{i1}(u_1) \dots A_{in}(u_n)},
\end{equation*}
with $q_s(u_1) \sim (p_1^+)^s$. Commuting charges means these can be simultaneously diagonalized. 

Consider particles as localized wave-packets:
\begin{equation*}
	\psi(x) \sim \int_{-\infty}^{\infty} \dd{p} e^{-a^2(p-p_1)^2} e^{i p (x - x_1)}.
\end{equation*}
An operator acting on $\psi$ gives momentum dependent phase factor 
\begin{equation*}
	\rightarrow \tilde{\psi} = \int_{-\infty}^{\infty} \dd{p} e^{-a^2(p-p_1)^2} e^{ip(x-x_1)} e^{-i\phi(p)},
\end{equation*}

Major contribution to integral close to $p_1$. Expand $\phi(p)$ and find modified values ($e^{-i\phi(p)} \simeq e^{-i\phi(p_1)} \cdot e^{-ip \phi'(p_1)}$)
\begin{equation*}
	\tilde{p}_1 = p_1, \tilde{x}_1 = x_1 + \phi'(p_1).
\end{equation*}
Position of particle $k$ is shifted by $\phi'(k)$.

Assume $Q_s \sim P^s$ and act with $e^{i\alpha Q_s}$, then $\phi_s(p) = \alpha p^s$. The particle with momentum $p_k$ is shifted by $s \alpha p_k^{s-1}$. ($s=1$: momentum operator $P$ shifts by constant.)
\end{proof}
